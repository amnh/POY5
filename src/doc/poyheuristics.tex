\section{Introduction}

As the level of phylogenetic analysis increases---from individual loci to chromosomes to genomes containing multiple chromosomes---so does computational complexity. In \poy, a significant increase in computational time results from combining in single process cladogram searching with co-optimiza\-tion of nucleotide pairwise alignments, rearrangements of loci within a chromosome, and rearrangements of chromosome fragments within the genome . As a result, a phylogenetic analysis involves a set of nested computationally ``hard'' (NP-complete) problems that makes finding the exact solution impossible. In addition, the increasing sequence length heterogeneity (at the levels of nucleotides, loci, and chromosomes) and the ever-growing sizes of datasets further contribute to computational complexity making it impossible to obtain an exact solution in a reasonable time.

To circumvent the problem computational intractability, and, hence, the speed of the analyses, \poy employs a battery of approximate, or heuristic, methods that function at different levels of analysis. As with all heuristic procedures, a tradeoff is involved: a substantial decrease in execution time comes at a price of obtaining possibly less accurate  and less precise results (however, the extent of the tradeoff is difficult to evaluate in the analyses of real large datasets). Therefore, it becomes important to understand the combined effect of different heuristic methods, so that the chosen search strategy balances the computational time with a ``reasonable'' accuracy of the result.

Here we provide general guidelines for using different heuristic methods, explore their combined effect, and suggest the choice of parameters that can be explored to provide the best result for specific cases. Real datasets differ greatly in size and complexity, so that no single optimal strategy can be suggested. These guidelines, however, should enable the investigator to design an efficient strategy that will tailor to the peculiarities of a given dataset.

In addition to heuristic methods, this chapter attempts to assist with the selection of transformation cost regimes. Alternative cost regimes can significantly affect the outcome of the analysis, that becomes particularly apparent in dealing with large, genome-level datasets, where multiple cost regimes are used simultaneously to specify transformations at different levels of analysis. Most difficulties stem from selecting the most reasonable combination of parameters that affect optimization of DNA sequence data at the levels of nucleotides, loci, and chromosomes.

\section{Data treatment}

Direct optimization (see \emph{Character optimization} section below) involves comparing all potential nucleotide homologies between two sequences. Consequently, the time it takes is proportional to the product of the lengths of the sequences compared. This procedure can be time consuming for long and greatly differing in length DNA fragments. In cases where unambiguous (such as long completely conserved regions) sequence fragments can be identified, partitioning the long sequences into smaller fragments delimited by these regions can significantly reduce computational time. Such economy is reached because nucleotide homologies are not examined over the separate partitions. This strategy assumes that the fragments are mutually exclusive and are putatively homologous across terminals.

At the level of nucleotides, individual fragments in a locus can be separated by the pound symbols (``\#'') or contained as individual files (that is, treated as partitions). When ``\#'' are used, their number must be the same across homologous sequences. Alternatively, the argument of \poyargument{auto\_sequence\_partition} of the command \ccross{transform}. At the chromosome level, individual loci can be separated by pipes (``$\vert$'').

\begin{center}
\begin{tabular}{| l  l  p{.35\textwidth}|}
	\hline
	Level of analysis & Heuristic & Implementation \\ \hline \hline
	Nucleotides & Fragment sequences & Manually separating fragments or use
	\poycommand{transform (auto\_sequence\_partition)}\\
	Locus & Fragment chromosome & Manually insert pipes separating loci \\
	Chromosomes & NA & NA \\
	\hline	
\end{tabular}
\end{center}

\section{Character optimization}
Minimizing overall cladogram cost is an NP hard problem dependent on the lowest cost assignment of HTU sequences.  POY implements direct optimization (DO; ~\cite{wheeler1996}) and fixed-states optimization (FSO; ~\cite{wheeler1999a}) heuristics to determine the set of HTU sequences comprising the internal nodes of each cladogram constructed.  Direct Optimization decomposes the problem into a series of two-node comparisons, calculating locally optimal solutions, which generates the total cladogram cost.  An advantage of direct optimization is that it allows for the exploration of a large diversity of putative homologies and selects the scheme that yields the most optimal solution. This is useful in analyzing sequences of different length, where site-to-site homologies are uncertain.  Because the procedure is based on a greedy algorithm, it requires multiple iterations (independent initial cladogram builds) and extensive tree searches to reach a potentially global minimum.  In contrast, fixed-states optimization does not calculate HTU sequences but rather optimizes those observed in terminal taxa. These internal node sequences then are diagnosed using dynamic programming based on a matrix of edit costs between sequences.  In the fixed-states implementation cladogram optimization is independent of sequence lengths, and as the number of sequences increase so to does the pool from which the HTU sequences are drawn, thereby improving cladogram cost estimation. Because of these properties fixed-states optimization is recommended as an initial approximation strategy for large data sets of variable length sequences.  

\begin{center}
\begin{tabular}{| l  l  p{.35\textwidth}|}
	\hline
Level of analysis&Heuristic&Implementation \\ \hline \hline
Nucleotides&DO&Default strategy\\
Nucleotides&FSO&\poycommand{transform(fixedstates)}\\
Loci&FSO&\poycommand{transform(dynamic\_pam:(approx))}\\
Chromosomes&NA&NA\\
\hline	
\end{tabular}
\end{center}

Further approximations and economies can be achieved by varying parameters of commands, such as selecting a limited subset of trees for subsequent analysis limiting the number of replicates, and examining intermediary results from an interrupted analysis.

\section{Tree searching}
The heuristic approaches to cladogram searching include random addition of taxa, branch swapping (TBR and SPR), simulated annealing (the ratchet and tree-drifting), and genetical algorithms (tree fusing). These techniques, frequently used in combination, allow a more efficient exploring of tree space and provide the means of finding more globally optimal solutions. These methods are widely used in phylogenetics \cite{felsenstein2004a, wheeleretal2006}, although \poy implements additional modifications of these procedures.

Typical search strategy in \poy involves consecutive application of tree search algorithms that begin with generating multiple, randomly selected starting points [Random Addition Sequences (RAS) or Wagner trees]. The importance of multiple starting trees cannot be overemphasized and a successful search shall maximize the number of RAS. However, making a tree search more exhaustive by increasing the number of starting trees comes at a price of longer computation time. Therefore, it is advised here to estimate the amount of time it takes to complete a single replicate and takes this information in consideration when designing a more exhaustive strategy. The  number of replicates used by \poy practitioners for datasets of moderate size (70-100 terminals) ranges from 100 to 250. Here are some examples of search strategies:
\begin{description}
\item[RAS+SPR/TBR+Ratchet] The strategy is for a thorough search for a data set of 100 or fewer taxa. A diversity of starting points is generated by multiple RAS, each refined by a local search (TBR or a combination of SPR and TBR, the latter is an efficient default strategy in \poy). Ratcheting is used to examine tree space that potentially has not been explored by the local searches.
\item[RAS+SPR/TBR+Ratchet+Tree Fusing]  Adding tree fusing step allows for combining the best sectors of cladograms that can potentially yield a tree of shorter length. Empirical studies showed that adding tree fusing after replicate rounds enhances the results only when dealing with data sets with more than 50 taxa.
\item[RAS+SPR/TBR+Ratchet+Tree Drifting+Tree Fusing] Tree Drifting can be used in place of or in addition to the Ratchet.
\item[Input Trees+SPR/TBR+Ratchet+Tree Drifting+Tree Fusing] For more exhaustive searches, the best trees obtained from the initial searches using the strategies outlines above, can be used as input trees for subsequent analyses. In doing so, the RAS step can be omitted because searching starts with trees approximating the globally optimal tree(s).
\end{description}

The aggressiveness of searches can be adjusted by varying parameters of the branch swapping, ratchet, tree fusing, and tree drifting commands.

Further economies can be reached by using a combination of different character optimization methods. For example, initial searches can be conducted under faster static approximation (that converts sequence data into static homology characters; see \emph{Character optimization} section), whereas the final refinement can be performed using direct optimization.

\section{Chromosome heuristics}
Analysis of chromosomal data requires heuristic procedures to estimate
rearrangement events in addition to nucleotide transformations. Chromosomal
data are divided into four different classes, namely breakinv, annotated, 
chromosome and genome. In the following we discuss the complexity of
each class.

\subsection{Breakinv character}
Breakinv character is the most simple form of chromosome data. Each breakinv
character (chromosome) is presented by a sequence of general alphabet characters each codes for
one gene. The transformation cost matrix among characters is calculated in advance
and provided by users. During the tree search, \emph{pairwise alignments with rearrangements} (PAR) 
between two breakinv characters is constructed.
The PAR generalizes the ordinary pairwise alignment by allowing rearrangements of character order. 
Since an exact solution for the PAR problem is likely intractable, we developed
a heuristic approach that is a compromise between computational expense and
alignment quality~\cite{vinh2006}. The method is comprised of two phases: first, it
creates an initial PAR using stepwise addition strategy; second, it improves the 
initial PAR by pairwise position swapping techniques. The second step is
repeated several iterations until either no improvement is found or the number
of swap iterations exceeds a user-defined maximum number,
\emph{swap\_med parameter}. 
The runtime complexity of the approaches is O($n^4 \times swap\_med$) where $n$ is 
the number of genes.

\begin{table}[t]
\caption{The influence of \emph{swap\_med} to running time and tree cost
         on a dataset containing 22 taxa with approximate 20 genes}
\label{swapMedComp} 
\begin{center}
\begin{tabular}{l c c}
\hline
	swap\_med & tree cost & time (seconds) \\
\hline
 	   0 		& 954 &   28 \\
 	   1 		& 948 &   52 \\
 	   2 		& 871 &   77 \\
 	   4 		& 882 &   97 \\
 	   8 		& 852 &   102 \\
\hline
\end{tabular}
\end{center}
\end{table}

Table \ref{swapMedComp} shows that the increase of \emph{swap\_med} 
results in increases the runtime. However, it does not guarantee the
improvement of tree cost. The \emph{swap\_med} default is one.


\subsection{Annotated chromosome character}
The annotated character type is a more general presentation of chromosome data 
than the breakinv character. Each annotated character consists of 
a sequence of loci/genes separated by pipes (`` $\vline$ ''). 
This data type allows for locus-level rearrangements as well as
nucleotide transformations. Locus homologies are
determined dynamically, but based on annotated regions~\cite{vinh2006}.
Given that two annotated characters each have $m$ genes, the pairwise alignment method
first calculates pairwise distances
among genes are calculated and then applies the algorithm to reconstruct
the PAR. 
The runtime complexity of the algorithm is O($m^2 \times l^2 + n^4 \times
swap\_med$) where $l$ is the average length of genes. Note that annotated
character does not require characters to have the same number of genes.
Although a large of number equally optimal PARs could be constructed
between two annotated characters, only a user-defined maximum number of PARs, \emph{median},  are kept
during the tree search. Our experience is that the increase of \emph{median}
does not usually result in the improvement of tree cost. The default value of
\emph{median} is one. 

It takes approximately 4 minutes to construct a Wagner tree of 10 taxa each contains
8 genes of length approximate 300 nucleotides. We conducted a 
SPR search on the constructed Wanger tree to examine the runtime and tree
improvement. SPR search takes approximately 13 minutes and reduces 
the tree cost about one percent.



\subsection{Chromosome character}

Unannotated chromosomal sequences are the most general presentation of
chromosome data where each chromosome consists of a long sequence of nucleotides.
To analyze this character type we developed an approach to construct a \emph{comprehensive chromosome pairwise
 alignment} that fulfills four conditions:
(1) all putative homologies among loci are determined automatically, 
(2) each locus is either aligned with only one putatively homologous locus or
considered as a locus indel, 
(3) loci are allowed to rearrange
(4) the total cost to transform one genome into another genome
(i.e. nucleotide transformation costs, locus indel costs, 
and locus order rearrangement costs) is minimized.  To this end, 
the approach consists of two phases. First, reliable homologies 
between two chromosomes are detected automatically. Second,
conserved areas serve as anchors to divide each chromosome into 
a sequence of separated loci.  To construct comprehensive chromosome
pairwise alignments the same method used for annotated chromosomes is applied~\cite{vinh2007}. 
Note that only a user-defined maximum number of PARs and medians are considered
during the tree search.


To find reliable homologies between two chromosomes, we apply a three-step algorithm.
First, identical segments, called \emph{seeds}, with lengths greater or equal to
user-defined \emph{seed\_length} between two genomes are detected using suffix tree
structure. Detected seeds whose distance is not greater than
\emph{rearranged\_len} are connected to construct larger conserved areas, called
\emph{blocks}. Blocks whose lengths are greater than the user-defined significant block length 
threshold \emph{sig\_block\_len} are considered as reliable homologies. 

A discussion of the influence of \emph{seed\_len},
\emph{sig\_block\_len} and \emph{rearranged\_len} follows:

The best default value of \emph{seed\_len} is \texttt{t}.
The higher the \emph{seed\_len} value, the fewer seeds are detected,
that, in turn influences the number of blocks recognized. Conversely, if the
value of seed\_length is low, an increased number of seeds and, consequently, a
greater number of short blocks are detected.

The default value of \emph{sig\_block\_len} is \texttt{100}. 
If the value of \emph{sig\_block\_len} is low, small-size rearrangements are allowed;
whereas if the value of \emph{sig\_block\_len} is high only large-size
rearrangements can be detected.

The \emph{rearranged\_len} parameter sets a threshold
value under which homologous blocks separated by non-homologous regions can be
considered as a single block. 
The default for this parameter is \texttt{100}.  Therefore, if two inferred
homologous blocks are separated by less than 100 nucleotides they will be
treated as a single block in calculating of rearrangement events.

Thus, the combination of parameters \poycommand{seed\_length}, 
\poycommand{sig\_block\_len}, 
and \poycommand{rearranged\_len} significantly influence the estimation of inferred rearrangements.

\begin{table}[t]
\caption{The influence of \emph{seed\_length} to the runtime and tree cost
or 11 corona viruses 27-32kb in length}
\label{seedLength} 
\begin{center}
\begin{tabular}{l c c}
\hline
	\emph{seed\_length} & tree cost & runtime (minutes) \\
\hline
         7             & 109085   & 16\\
         9 (default)   & 91099   & 12\\
         11            & 91524   & 13\\
\hline
\end{tabular}
\end{center}
\end{table}

To examine the running time, we collect 11 corona viruses 
27-13kb in length. The program takes about 12 minutes
to reconstruct a Wagner tree, and around one hour 
for SPR swapping.



\begin{table}[t]
\caption{The default and suggested values of different parameters for chromosome
characters}
\label{defaultPam} 
\begin{center}
\begin{tabular}{l c c}
\hline
	name & default value cost & suggested value \\
\hline
    \poycommand{seed\_length}     & 9   & 5-15\\
    \poycommand{sig\_block\_len}  & 100 & 60-150\\
    \poycommand{rearranged\_len}   & 100 & 50-1000\\
    \poycommand{breakpoint}       &10   & 10-50\\
    \poycommand{inversion}        &none & 15-35\\
    \poycommand{approx}           &false& large data sets\\
    \poycommand{median}           &1    & 1-2\\
    \poycommand{swap\_med}        &1    & 1-2\\
    \poycommand{locus\_indel}     &opening 10, extension 1  & opening 10, extension 1\\
\hline
\end{tabular}
\end{center}
\end{table}


Table \ref{defaultPam} summarizes the default and suggested values of different
parameters for chromosome characters.


\section{Transformation cost regimes}
In analyses at the level of nucleotides, there are three general approaches to selecting transformation cost regimes most commonly used by \poy practitioners.
\begin{description}
\item[Equal costs] This approach assigns the same cost to all substitutions and indels, and does not take into account gap extension cost. For rationale for using this cost regime see Frost et al. \cite{frost2001} %and for other examples of its application see.
\item[Homology maximization] This approach, developed by De Laet \cite{delaet2005}, assigns costs \texttt{2, 3, and 1} to transformations, gap opening, and gap extension respectively. %For examples using this methods see
\item[Parameter sensitivity analysis] This method, suggested by Wheeler \cite{wheeler1995}, explores the effect of varying transformation costs by comparing results of analyses conducted under different cost regimes. Partition inconguence can subsequently be computed  for each cladogram and the parameter set that minimizes incongruence is selected as optimal. %For examples using this methods see
\end{description}
More specifically, it depends on relative costs of nucleotide- and locus-level transformations. Nucleotide-level transformations are specified by tcm argument, the locus-level rearrangements are specified by locus\_breakpoint or inversion costs. If locus\_level rearrangement costs are extremely high, the rearrangements are not going to be counted. On the other hand, if their cost is very low (equal or slightly above that of the nucleotide-level rearrangements), rearrangements can be frequent (depending on the seed\_block\_len and seed\_length settings).

When DNA sequence data is combined with morphological data, the cost for morphological character transformations is customarily is set to be the same as for substitutions.

