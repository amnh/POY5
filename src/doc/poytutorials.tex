These tutorials are intended to provide guidance for more sophisticated applications of \poy that involve 
multiple steps and a combination of different commands. Each tutorial contains a \poy script that is followed 
by detailed commentaries explaining the rationale behind each step of the analysis. Although these analyses 
can be conducted interactively using the \emph{Interactive Console} or running separate sequential analyses
 using the \emph{Graphical User Interface}, the most practical way to do this is to use \poy scripts (see 
 \emph{ POY5 Quick Start} for more information on \poy scripts).\\

{\bf It is important to remember that the numerical values for most command arguments will differ substantially 
depending on type, complexity, and size of the data. Therefore, the values used here should not be taken as 
optimal parameters.}\\

The tutorials use sample datasets that are provided with \poy installation but can also be downloaded from 
the \poy site at
\begin{center}
\url{http://research.amnh.org/scicomp/projects/poy.php}
\end{center}
The minimally required items to run the tutorial analyses are the \poy application and the sample data files. 
Running these analyses requires some familiarity with the \poy interface and command structure that can be 
found in the preceding chapters.

\section{Combining  search strategies}{\label{tutorial1}}
The following script implements a strategy for a thorough search. This is accomplished by generating a large 
number of independent initial trees by random addition sequence and combining different search strategies 
that aim at thoroughly exploring local tree space and escape the effect of composite optima by effectively 
traversing the tree space. In addition, this script shows how to output the status of the search to a log file and
 calculate the duration of the search. 

\begin{verbatim}
(* search using all data *)
read("9.fas","31.ss", aminoacids:("41.aa"))
(* We select as root the taxon with name t1. If we wanted
the taxon Locusta_migratoria, we would write:
root:"Locusta_migratoria" *)
set(seed:1,log:"all_data_search.log",root:"t1")
report(timer:"search start")
transform(tcm:(1,2),gap_opening:1)
build(250)
swap(threshold:5.0)
select(unique)
perturb(transform(static_approx),iterations:15,ratchet:(0.2,3))
select()
fuse(iterations:200,swap())
select()
report("all_trees",trees:(total),"constree",graphconsensus,
"diagnosis",diagnosis)
report(timer:"search end")
set(nolog)
exit()
\end{verbatim}

\begin{itemize}
\item \texttt{(* search using all data *)} This first line of the script is a comment. While comments are optional 
and do not affect the analyses, they are useful for housekeeping purposes.
\item \texttt{read("9.fas","31.ss", aminoacids:("41.aa"))}
This command imports all the nucleotide sequence data files (all files with the extension \texttt{.seq}), a 
morphological data file \texttt{morph.ss} in Hennig86 format, and an aminoacid data file \texttt{myosin.aa}.
\item \texttt{set(seed:1,log:"all\_data\_search.log",root:"t1")} The \poycommand{set} command specifies 
conditions prior to tree searching. The \poyargument{seed} is used to ensure that the subsequent 
randomization procedures (such as tree building and selecting) are reproducible. Specifying the log 
produces a file, \texttt{all\_data\_search.log} that provides an additional means to monitor the process 
of the search. The outgroup (\texttt{taxon1}) is designated by the \poyargument{root}, so that all the 
reported trees have the desired polarity. By default, the analysis is performed using direct optimization.
\item \texttt{report(timer:"search start")} In combination with \texttt{report(timer:\\"search end")}, this commands 
reports the amount of time that the execution of commands enclosed by \poyargument{timer} takes. In this 
case, it reports how long it takes for the entire search to finish. Using timer is useful for planning a complex 
search strategy for large datasets that can take a long time to complete: it is instructive, for example, to know 
how 
long a search would last with a single replicate (one starting tree) before starting a search with multiple 
replicates.
\item \texttt{transform(tcm:(1,2),gap\_opening:1)} This command sets the \\transformation cost matrix for 
molecular data to be used in calculating the cost of the tree. Note, that in addition to the substitution and 
indel costs, the \poycommand{transform} specifies the cost for gap opening.
\item \texttt{build(250)} This commands begins tree-building step of the search that generates 250 
random-addition trees. A large number of independent starting points insures that a large portion of tree 
space will be examined.
\item \texttt{swap(threshold:5.0)} \poycommand{swap} specifies that each of the 250 trees is subjected to 
alternating SPR and TBR branch swapping routine (the default of \poy). In addition to the most optimal trees, 
all the suboptimal trees found within 5\% of the best cost are thoroughly evaluated. This step ensures that 
the local searches settled on the local optima.
\item \texttt{select(unique)} Upon completion of branch swapping, this command retains topologically unique trees.
Contra \texttt{select()}, which selects topologically \emph{unique} and \emph{optimal} trees,  \texttt{select(unique)} 
selects \emph{all} unique trees, regardless of cost, thus ensuring that a larger tree space is explored.
\item \texttt{perturb(transform(static\_approx),iterations:15,ratchet:\\(0.2,3))} This command subjects the 
resulting trees to 15 rounds of ratchet, re-weighting 20\% of characters by a factor of 2. During ratcheting, 
the dynamic homology characters are transformed into static homology characters, so that the fraction of 
nucleotides (rather than of sequence fragments) is being re-weighted. This step, that begins at multiple 
local maxima, is intended to further traverse the tree space in search of a global optimum.
\item \texttt{fuse(iterations:200,swap())} In this step, up to 200 swaps of subtrees identical in terminal 
composition but different in topology, are performed between pairs of best trees recovered in the previous 
step. This is another strategy for further exploration of tree space. Each resulting tree is further refined by 
local branch swapping under the default parameters of \poycommand{swap}.
\item \texttt{select()} Upon completion of branch swapping, this command retains only optimal and 
topologically unique trees; all other trees are discarded from memory.
\item \texttt{report("all\_trees",trees:(total),"constree",graphconsensus,\\"diagnosis",diagnosis)} This command 
produces a series of outputs of the results of the search. It includes a file containing best trees in parenthetical 
notation and their costs (\texttt{all\_trees}), a graphical representation (in PDF format) of the strict consensus 
(\texttt{constree}), and the diagnoses for all best trees (\texttt{diagnosis}).
\item \texttt{report(timer:"search end")} This command stops timing the duration of search, initiated by the 
command \texttt{report(timer:"search start")}.
\item \texttt{set(nolog)} This command stops reporting any output to the log file, \texttt{all\_data\_search.log}.
\item \texttt{exit()} This commands ends the \poy session.
\end{itemize}

\section{Searching under iterative pass}{\label{tutorial2}}
The following script implements a strategy for a thorough search under iterative pass optimization. The iterative
 pass optimization is a very time consuming procedure that makes it impractical to conduct under this kind of 
optimization (save for very small datasets that can be analyzed within reasonable time). The iterative pass, 
however, can be used for the most advanced stages of the analysis for the final refinement, when a potential 
global optimum has been reached through searches under other kinds of optimization (such as direct 
optimization). Therefore, this tutorial begins with importing an existing tree (rather than performing tree building
 from scratch) and subjecting it to local branch swapping under iterative pass.

\begin{verbatim}
(* search using all data under ip *)
read("9.fas","31.ss",aminoacids:("41.aa"))
read("inter_tree.tre")
transform(tcm:(1,2),gap_opening:1)
set(iterative:approximate:2)
swap(around)
select()
report("all_trees",trees:(total),"constree",graphconsensus,
"diagnosis",diagnosis)
exit()
\end{verbatim}

\begin{itemize}
\item \texttt{(* search using all data under ip *)} This first line of the script is a comment. While comments are 
optional and do not affect the analyses, they are useful for housekeeping purposes.
\item \texttt{read("9.fas","31.ss",aminoacids:("41.aa"))} This command imports all the nucleotide sequence 
data files (all files with the extension \texttt{.seq}), a morphological data file \texttt{morph.ss} in Hennig86 format, 
and an aminoacid data file \texttt{myosin.aa}.
\item \texttt{read("inter\_tree.tre")} This command imports a tree file, \texttt{inter\_tree.tre}, that contains the most 
optimal tree from prior analyses. 
\item \texttt{transform(tcm:(1,2),gap\_opening:1)} This command sets the transformation cost matrix for molecular 
data to be used in calculating the cost of the tree. Note, that in addition to the substitution and indel costs, the 
\poycommand{transform} specifies the cost for gap opening.
\item \texttt{set(iterative:approximate:2)} This command sets the optimization procedure
    to iterative pass such that approximated three dimensional alignments generated using pairwise alignments 
    will be considered.  The program will iterate either two times, or until no further tress cost improvements can 
    be made.
\item \texttt{swap(around)} This commands specifies that the the imported tree is subjected to alternating SPR 
and TBR branch swapping routine (the default of \poy) following the trajectory of search that completely 
evaluates the neighborhood of the tree (by using \poyargument{around}).
\item \texttt{select()} Upon completion of branch swapping, this command retains only optimal and topologically 
unique trees; all other trees are discarded from memory.
\item \texttt{report("all\_trees",trees:(total),"constree",\\graphconsensus,"diagnosis",diagnosis)} This command 
produces a series of outputs of the results of the search. It includes a file containing best trees in parenthetical 
notation and their costs (\texttt{all\_trees}), a graphical representation (in PostScript format) of the strict 
consensus (\texttt{constree}), and the diagnoses for all best trees (\texttt{diagnosis}).
\item \texttt{exit()} This commands ends the \poy session.
\end{itemize}

\section{Bremer support}{\label{tutorial 3}}

This tutorial builds on the previous tutorials to illustrate Bremer support 
calculation on trees constructed using dynamic homology characters
    
   \begin{verbatim}
(* Bremer support part 1: generating trees *)
read("18s.fas",28s.fas")
set(root:"Americhernus")
build(200)
swap(all,visited:"tmp.trees", timeout:3600)
select()
report("my.tree",trees)
exit()

(* Bremer support part 2: Bremer calculations *)
read("18s.fas",28s.fas","my.tree")
report("support_tree.pdf",graphsupports:bremer:"tmp.trees")
exit()
\end{verbatim}

\begin{itemize}
\item \texttt{(* Bremer support part1: generating trees *)} This first line of the script is a comment. While comments
 are optional and do not affect the analyses, they are useful for housekeeping purposes. 
\item \texttt{read("18s.fas","28s.fas")} This command imports the nucleotide sequence files \texttt{18s.fas, 
28s.fas}.
\item \texttt{set(root:"Americhernus")} The \poycommand{set} command specifies conditions prior to tree 
searching. The outgroup (\texttt{Americhernus}) is designated by the \poyargument{root}, so that all the reported 
trees have the desired polarity.     
\item \texttt{build(200)} This command initializes tree-building and generates 200 random-addition trees.      
\item \texttt{swap(all,visited:"tmp.trees", timeout:3600)} The \poycommand{swap} command specifies that each of 
the trees be subjected to an alternating SPR and TBR branch swapping routine (the default of \poy).  The 
\poyargument{all} argument turns off all swap heuristics. The \poyargument{visited:"tmp.trees"} argument stores 
every visited tree in the file specified.  Although the visited tree file is compressed to accommodate the large 
number of trees it will accumulate, the argument \poyargument{timeout} can be used to limit the number of 
seconds allowed for swapping also limiting the size of the file.  Alternately  the  \poycommand{swap} command 
can be performed as a separate analysis and terminated at the users discretion to maximize the number of trees 
generated.
\item \texttt{select()} Upon completion of branch swapping, this command retains only optimal and topologically
 unique trees; all other trees are discarded from memory. 
\item \texttt{report("my.tree",trees)} This command will save the swapped tree, \\ \texttt{my.tree} to a file. 
\item \texttt{exit()} This commands ends the \poy session.

\item \texttt{(* Bremer support part 2: Bremer calculations *)}  A comment indicating the intent of the commands 
which follow.
\item \texttt{read("18s.fas","28s.fas","my.tree")} This command imports the nucleotide sequence files 
\texttt{18s.fas, 28s.fas} and the tree file, \texttt{my.tree} for which the support values will be generated.  It is 
important to only read the selected \texttt{"my.tree"} file rather than the expansive  \texttt{"tmp.trees"} file which 
will be used in bremer calculations.
\item \texttt{report("support\_tree.pdf",graphsupports:bremer:"tmp.trees")} \\The \poycommand{report} command 
in combination with a file name and the \\ \poyargument{graphsupports} generates a pdf file designated by the 
name \texttt{support\_tree.pdf} with bremer values for the selected trees held in \texttt{tmp.trees}.  It is strongly 
recommended that this more exhaustive approach is used for calculating Bremer supports rather than simply 
using the \\ \poyargument{graphsupports} defaults.  
\item \texttt{exit()} This commands ends the \poy session.
\end{itemize}

\section{Jackknife support}{\label{tutorial 4}}

This tutorial illustrates calculating Jackknife support values for trees constructed with static homology characters.  
Although it is possible to calculate Jackknife and Bootstrap support values 
for trees constructed using dynamic homology characters, it is not recommended because resampling of 
dynamic characters occurs at the fragment (rather than nucleotide) level. Alternately dynamic homology 
characters can be converted to static characters using the transform argument \poyargument{static\_approx}.  
Character transformation, however, may result in a discrepancy between tree costs generated using dynamic 
homology characters and those generated using static homology characters. Therefore after such a data 
transformation is performed, an extra round of swapping is recommended to insure that the local minimum tree 
length is reached for the static homology characters prior to calculating support values  support values.

\begin{verbatim}
(* Jackknife support for static homology trees *)
read(prealigned:("28s.aln",tcm:(1,2)))
set( root:"Americhernus")
build()
swap()
select()
calculate_support(jackknife:(remove:0.50,resample:1000), 
build(5),swap(tbr,trees:3))
report("jacktree",graphsupports)
exit()
\end{verbatim}

\begin{itemize}
\item \texttt{(* Jackknife support for static homology trees *)} This first line of the script is a comment. While
 comments are optional and do not affect the analyses, they are useful for housekeeping purposes.
\item \texttt{read(prealigned:("28s.aln",tcm:(1,2)))} This command imports the prealigned nucleotide sequence 
file \texttt{28s.aln}, and treats the characters as static with the prescribed transformation cost matrix.
\item \texttt{set(root:"Americhernus")} The \poycommand{set} command specifies conditions prior to tree 
searching. The outgroup (\texttt{Americhernus}) is designated by the \poyargument{root}, so that all the reported 
trees have the desired polarity.     
\item \texttt{build()} This command begins the tree-building step of the search that generates by default 10 
random-addition trees. It is essential that trees are either specified from a file or that trees are built and loaded in 
memory before attempting to calculate support values.
\item \texttt{swap()} The \poycommand{swap} command specifies that each of the trees be subjected to an 
alternating SPR and TBR branch swapping routine (the default of \poy).
\item \texttt{select()} Upon completion of branch swapping, this command retains only optimal and topologically 
unique trees; all other trees are discarded from memory. 
\item \texttt{calculate\_support(jackknife,(remove:0.50,resample:1000)} The \poycommand{calculate\_support} 
command generates support values as specified by the \poyargument{jackknife} argument for each tree held in 
memory. During each pseudoreplicate half of the characters will be deleted as specified in the argument
\poyargument{remove:0.50}. 
\item \texttt{report("jacktree",graphsupports:jackknife)}  The \poycommand{report} command in combination with 
a file name and the \poyargument{graphsupports} generates a pdf file with jackknife values designated by the 
name specified (\emph{i.e.} \texttt{jacktree}). 
\item \texttt{exit()} This commands ends the \poy session.
\end{itemize}

\section{Sensitivity analysis}{\label{tutorial 5}}

This tutorial demonstrates how data for parameter sensitivity analysis is generated. Sensitivity analysis 
\cite{wheeler1995} is a method of exploring the effect of relative costs of substitutions (transitions and 
transversions) and indels (insertions and deletions), either with or without taking gap extension cost into 
account. The approach consists of multiple iterations of the same search strategy under different parameters, 
(\emph{i.e} combinations of substitution and indel costs. Obviously, such analysis might become time 
consuming and certain methods are shown here how to achieve the results in reasonable time. This tutorial also
 shows the utility of the command \poycommand{store} and how transformation cost matrixes are imported and 
 used.

\poy does not comprehensively display the results of the sensitivity analysis or implements the methods to select 
a parameter set that produces the optimal cladogram, but the output of a \poy analysis (such as the one 
presented here) generates all the necessary data for these additional steps.

For the sake of simplicity, this script contains commands for generating the data under just two parameter  sets. 
Using a larger number of parameter sets can easily be achieved by replicating the repeated parts of the script 
and substituting the names of input cost matrixes.

\begin{verbatim}
(* sensitivity analysis *)
read("9.fas")
set(root:"t1")
store("original_data")
transform(tcm:("111.txt"))
build(100)
swap(timeout:3600)
select()
report("111.tre",trees:(total) ,"111con.tre",consensus,
"111con.pdf",graphconsensus)
use("original_data")
transform(tcm:"(112.txt"))
build(100)
swap(timeout:3600)
select()
report("112.tre",trees:(total),"112con.tre",consensus,
"112con.pdf",graphconsensus)
exit()
\end{verbatim}

\begin{itemize}
\item \texttt{(* sensitivity analysis *)} This first line of the script is a comment. While comments are optional and do not
 affect the analyses, they are useful for housekeeping purposes.
\item \texttt{read("9.fas")} This command imports all dynamic homology nucleotide data.
\item \texttt{set(root:"t1")} The outgroup (\texttt{taxon1}) is designated by the \poyargument{root}, so that all the 
reported trees have the desired polarity.
\item \texttt{store("original\_data")} This commands stores the current state of analysis in memory in a temporary file, 
\texttt{original\_data}.
\item \texttt{transform(tcm:"(111.txt"))} This command applies a transformation cost matrix from the file \texttt{111.txt} to 
for subsequent tree searching.
\item \texttt{build(100)} This commands begins tree-building step of the search that generates 250 random-addition 
trees. A large number of independent starting point insures that thee large portion of tree space have been 
examined.
\item \texttt{swap(timeout:3600)} \poycommand{swap} specifies that each of the 100 trees build in the previous step is 
subjected to alternating SPR and TBR branch swapping routine (the default of \poy). The argument 
\poyargument{timeout} specifies that 3600 seconds are allocated for swapping and the search is going to be stopped 
after reaching this limit. Because sensitivity analysis consists of multiple independent searches, it can take a 
tremendous amount of time to complete each one of them. In this example, \poyargument{timeout} is used to prevent 
the searches from running too long. Using \poyargument{timeout} is optional and can obviously produce suboptimal 
results due to insufficient time allocated to searching. A reasonable timeout value can be experimentally obtained by 
the analysis under one cost regime and monitoring time it takes to complete the search (using the argument 
\poyargument{timer} of the command \poycommand{set}). The advantage of using \poyargument{timeout} is saving 
time in cases where a local optimum is quickly reached and the search is trapped in its neighborhood.
\item \texttt{select()} Upon completion of branch swapping, this command retains only optimal and topologically 
unique trees; all other trees are discarded from memory.
\item \texttt{report("111.tre",trees:(total) ,"111con.tre",consensus,\\"111con.pdf", graphconsensus)} This command 
produces a file containing best tree(s) in parenthetical notation and their costs (\texttt{111.tre}), a a file containing the 
strict consensus in parenthetical notation \\(\texttt{111con.tre}), and a graphical representation (in PDF format) of the 
strict consensus (\texttt{111con.pdf}).
\item \texttt{use("original\_data")} This command restored the original (non-trans\-formed) data from the temporary file 
\texttt{original\_data} generated by \poycommand{store}.
\item \texttt{transform(tcm:("112.txt"))} This command applies a different transformation cost matrix from the file 
\texttt{112.txt} to for another round of tree searching under this new cost regime.
\item \texttt{build(100)} This commands begins tree-building step of the search that generates 100 random-addition 
trees. A large number of independent starting point insures that thee large portion of tree space have been 
examined.
\item \texttt{swap(timeout:3600)} \poycommand{swap} specifies that each of the 100 trees build in the previous step is 
subjected to alternating SPR and TBR branch swapping routine (the default of \poy) to be interrupted after 3600 
seconds (see the description in the previous iteration of the command above).
\item \texttt{select()} Upon completion of branch swapping, this command retains only optimal and topologically 
unique trees; all other trees are discarded from memory.
\item \texttt{report("112.tre",trees:(total),"112con.tre",consensus,\\"112con.pdf", graphconsensus)} This command 
produces a set of the same kinds of outputs as generated during the first search (see above) but under a new cost 
regime.
\item \texttt{exit()} This commands ends the \poy session.
\end{itemize}

\section{Chromosome analysis: unannotated sequences}{\label{tutorial 6}}

This tutorial illustrates the analysis of chromosome-level transformations using 
unannotated sequences, i.e., contiguous strings of sequences without prior 
identification of independent regions. 

\begin{verbatim}
(* Chromosome analysis of unannotated sequences *)
read(chromosome:("mit5.txt"))
transform(chromosome:(locus_inversion:15,locus_indel:(10,0.9)))
transform(chromosome:(annotate:(mauve,25.0,0.3,0.01,0.08)))
transform(fixed_states:("mauveout",
_polymorphism))
build()
swap()
select()
report("chrom",diagnosis)
report("consensustree",graphconsensus)
transform(fixed_states:("mauveout",ignore_polymorphism))
exit()
\end{verbatim}

\begin{itemize}
\item \texttt{(* Chromosome analysis of unannotated sequences *)} This first line of the script is a comment. While 
comments are optional and do not affect the analyses, they are useful for housekeeping purposes.
\item \texttt{read(chromosome:("mit5.txt"))} This command imports the unannotated chromosomal sequence file 
\texttt{mit5.txt}. The argument \poyargument{chromosome} specifies the characters as unannotated chromosomes.
\item \texttt{transform(chromosome:(locus\_inversion:15,locus\_indel:\\(10,0.9))}.  The \poycommand{trans\-form} 
followed by the argument \poyargument{chromosome} signifies the conditions to be applied when calculating 
chromosome-level (medians).  The argument \poyargument{locus\_inversion:15} applies a inversion distance 
between chromosome loci with the integer value determining the rearrangement cost. The argument 
\poyargument{locus\_indel:10,0.9} specifies the indel costs for the chromosomal segments, whereby the integer 10 
sets the gap opening cost and the float 0.9 sets the gap extension cost.
\item \texttt{transform(chromosome:(annotate:(mauve,25.0,0.3,0.01,0.08)))} The argument \poyargument{annotate:
(mauve)} specifies that the program will use the Mauve aligner \cite{darlingetal2004} to determine locally collinear 
homologous blocks within the chromosomal sequences.  The values that follow the Mauve option set the parameters 
for determining the lcb homologies: quality, coverage, and minimum and maximum lcb length relative to overall 
sequence length. In this case the lcb quality parameter which represents the cost of the lcb divided by its length of lcb 
is set to the relatively low value of 25 to facilitate the detection of blocks within the sequences.  The higher the lcb 
quality values will result in more stringent lcb determination and likely fewer local collinear blocks recovered.  The 
second parameter within the argument \poyargument{annotate:(mauve)} sets the minimum lcb sequence coverage at 
30\% meaning that if total length of an input sequence is, for example,100, a minimum coverage of 0.30 would require 
a the total length of all lcbs to be at least 30. The default value of 0.01 or 1\% is sets the minimum length of a given lcb 
relative to the length of the entire sequence (e.g. 100 for a 10,000 nucleotide sequence). The maximum length 
allowed for an lcb in this example is set at 8\% of the length of the total sequence.
\item \texttt{transform(fixed\_states:``mauveout'',ignore\_polymorphism))} This arguments for a MAUVE-based fixed\_states 
analysis reads input of unannotated chromosomes and uses the MAUVE algorithm to create an annotation that is then used to 
create the distance between two chromosomes. Here, MAUVE genome alignment files will be generated with the names 
``mauveout\_i\_j.alignment'' where i and j are median states. Sequence ambiguities will not be resolved to generate additional 
medians beyond those determined by the data.
\item \texttt{build()} This commands begins the tree-building step of the search that generates by default 10 random-
addition trees. It is highly recommended that a greater number of Wagner builds be implemented when analyzing 
data for purposes other than this demonstration.
\item \texttt{swap()} The \poycommand{swap} command specifies that each of the trees be subjected to an alternating 
SPR and TBR branch swapping routine (the default of \poy).
\item \texttt{select()} Upon completion of branch swapping, this command retains only optimal and topologically 
unique trees; all other trees are discarded from memory. 
\item \texttt{report("chrom",diagnosis)}  The \poycommand{report} command in combination with a file name and the 
\poyargument{diagnosis} outputs the optimal median states and edge values to a specified file (\texttt{chrom}). 
\item \texttt{report("consensustree",graphconsensus)}  The \poycommand{report} command in combination with a file 
name and the \poyargument{graphconsensus} generates a pdf strict consensus file of the trees generated 
(\texttt{consensustree}). 
\item \texttt{transform(fixed\_states:("mauveout", ignore\_polymorphism))}  \\The \poycommand{transform} command 
in combination with \poyargument{fixed\_states:\\("mauveout",ignore\_polymorphism))} is used to produce output that 
can be read into Mauve to generate a pairwise alignment tracking the movement of lcbs between sequences. 
Sequence ambiguities will not be resolved to generate additional medians beyond those determined by the data.
\item \texttt{exit()} This commands ends the \poy session.
\end{itemize}

\section{Chromosome analysis: annotated sequences}{\label{tutorial 7}}

This tutorial illustrates the analysis of chromosome-level transformations using 
annotated sequences, i.e., contiguous strings of sequences with prior 
identification of independent regions delineated by pipes  \texttt{"|"}. 

\begin{verbatim}
(* Chromosome analysis of annotated sequences *)
read(annotated:("aninv2"))
transform(chromosome:(locus_inversion:20,locus_indel:(10,1.5),
circular:(false,median:1,swap_med:1)))
build(100)
swap()
select()
report("Annotated",diagnosis)
report("consensustree",graphconsensus)
exit()
\end{verbatim}

\begin{itemize}
\item \texttt{(* Chromosome analysis of annotated sequences  *)} This first line of the script is a comment. While 
comments are optional and do not affect the analyses, they are useful for housekeeping purposes.
\item \texttt{read(annotated:("aninv2"))} This command imports the annotated chromosomal sequence file 
\texttt{aninv2}. The argument \poyargument{annotated} specifies the characters. 
\item \texttt{transform((chromosome:(locus\_inversion:20,locus\_indel:\\(10,1.5),median:1,swap\_med:1)))}  The 
\poycommand{transform} follow\-ed by the argument \poyargument{chromosome} specifies the conditions to be 
applied when calculating chromosome-level (medians).  The argument \poyargument{locus\_inversion:20} applies 
an inversion distance between chromosome loci with the integer value determining the rearrangement cost and 
using the default Caprara median solver. The argument \poyargument{locus\_indel:\\(10,1.5} specifies the indel costs 
for chromosomal segments, where the integer 10 sets the gap opening cost and the float 1.5 sets the gap extension 
cost.  The default values are applied to the arguments \poyargument{circular}  \poyargument{median} and 
\poyargument{swap\_med} arguments to minimize the time require for these nested search options.   To more 
exhaustively perform these calculations, trees generated from initial builds can be imported to the program and 
reevaluated with values greater than 1 entered for the \poyargument{median} and \poyargument{swap\_med} 
arguments.
\item \texttt{build()} This commands begins the tree-building step of the search that generates by 100 random-
addition trees.  It is highly recommended that a greater number of Wagner builds be implemented when analyzing 
data for purposes other than this demonstration.
\item \texttt{swap()} The \poycommand{swap} command specifies that each of the trees be subjected to an alternating 
SPR and TBR branch swapping routine (the default of \poy).
\item \texttt{select()} Upon completion of branch swapping, this command retains only optimal and topologically 
unique trees; all other trees are discarded from memory. 
\item \texttt{report("Annotated",diagnosis)}  The \poycommand{report} command in combination with a file name and 
the \poyargument{diagnosis} outputs the optimal median states and edge values to a specified file 
(\texttt{Annotated}). 
\item \texttt{exit()} This commands ends the \poy session.
\end{itemize}

%\section{Custom alphabet and break inversion characters}{\label{tutorial 8}}

%This tutorial illustrates the analysis of the break inversion character type.  Break inversion characters are generated 
%by transforming user-defined \poyargument {custom\_alphabet} characters.  
%For example, observations of developmental stages could be represented in a corresponding array such that for 
%each terminal taxon there is a sequence of observed developmental stages which are represented by a user-defined
% alphabet.  To allow rearrangement as well as indel events to be considered among alphabet elements, requires 
% transforming the \poyargument {custom\_alphabet} sequences to \poyargument {breakinv} characters. 
%
%\begin{verbatim}
%(* Custom Alphabet to Breakinv characters *)
%read(custom_alphabet:("ca1.fas","m1.fas"))
%transform(custom_to_breakinv:())
%transform(breakinv:(median_solver: siepel, locus_inversion:20, 
%locus_indel:(10,1.5),median:1,swap_med:1))
%build()
%swap()
%select()
%report("breakinv",diagnosis)
%report("consensustree",graphconsensus)
%exit()
%\end{verbatim}
%
%\begin{itemize}
%\item \texttt{(* Custom Alphabet to Breakinv characters  *)} This first line of the script is a comment. While comments 
%are optional and do not affect the analyses, they are useful for housekeeping purposes.
%\item \texttt{read(custom\_alphabet:("ca1.fas","m1.fas"))} This command imports the user-defined \poyargument 
%{custom\_alphabet} character file \texttt{ca1.fas} and the accompanying transformation matrix \texttt{m1.fas}.
%\item \texttt{transform(custom\_to\_breakinv)} This command transforms \poyargument {custom\_alphabet} characters 
%to \poyargument {breakinv} characters which allow for rearrangement operations.
%\item \texttt{transform(breakinv:(locus\_inversion:10,median\_solver:siepel, locus\_inversion:20, locus\_indel:
%(10,1.5),median:1,swap\_med:1))}  The \poycommand{transform} followed by the argument \poyargument{breakinv} 
%specifies the conditions to be applied when calculating medians. The argument \poyargument{[median
%\_solver:siepel} specifies that the Siepel median from the GRAPPA software package \cite{baderetal2002} will be 
%employed.  The argument \poyargument{locus \_inversion:20} applies an inversion rearrangement cost of 20 for 
%\poyargument {breakinv} elements. The argument \poyargument{locus\_indel:10,1.5} specifies the indel costs for 
%each \poyargument {breakinv} element, whereby the integer 10 sets the gap opening cost and the float 1.5 sets the 
%gap extension cost.  The default values are applied to the \poyargument{median} and \poyargument{swap\_med} 
%arguments to minimize the time require for these nested search options.   To more exhaustively perform these 
%calculations trees generated from initial builds can be imported to the program and reevaluated with values greater 
%than 1 designated for the \poyargument{median} and \poyargument{swap\_med} arguments.
%\item \texttt{build()} This commands begins the tree-building step of the search that generates by default 10 random-
%addition trees.  It is highly recommended that a greater number of Wagner builds be implemented when analyzing 
%data for purposes other than this demonstration.
%\item \texttt{swap()} The \poycommand{swap} command specifies that each of the trees be subjected to an alternating 
%SPR and TBR branch swapping routine (the default of \poy).
%\item \texttt{select()} Upon completion of branch swapping, this command retains only optimal and topologically 
%unique trees; all other trees are discarded from memory. 
%\item \texttt{report ("breakinv",diagnosis)}  The \poycommand{report} command in combination with a file name and 
%the \poyargument{diagnosis} outputs the optimal median states and edge values to a specified file (\texttt{breakinv}). 
%\item \texttt{exit()} This commands ends the \poy session.
%\end{itemize}

\section{Genome analysis: multiple chromosomes}{\label{tutorial 8}}

This tutorial illustrates the analysis of genome-level transformations using data from multiple chromosomes. 

\begin{verbatim}
(* Genome analysis of multiple chromosomes *)
read (genome:("gen5bp"))
transform(genome:(translocation:50))
transform(chromosome:(annotate:(mauve,25.0,0.3,0.01,0.08)))
transform(chromosome:(locus_breakpoint:80,locus_indel:(15,2.5),
median:1,swap_med:1))
build()
swap()
select()
report("genome",diagnosis)
report("genconsensus",graphconsensus)
exit()
\end{verbatim}

\begin{itemize}
\item \texttt{(* Genome analysis of multiple chromosomes*)} This first line of the script is a comment. While 
comments are optional and do not affect the analyses, they provide are useful for housekeeping purposes.
\item \texttt{read(genome:("gen5bp"))} This command imports the genomic sequence file \texttt{mit5.txt}. The 
argument \poyargument{genome} specifies the characters as data consisting of multiple chromomsomes.
\item \texttt{transform(genome:(translocation:50))} sets the breakpoint cost for the movement of lcbs from one 
chromosomal segment to another. 
\item \texttt{transform(chromosome:(annotate:(mauve,25.0,0.3,0.01,0.08)))} \\The argument \poyargument{annotate:
(mauve)} specifies that the program will use the Mauve aligner \cite{darlingetal2004} to determine locally collinear 
homologous blocks within the chromosomal sequences.  The values that follow the Mauve option set the parameters 
for determining the lcb homologies: quality, coverage, and minimum and maximum lcb length relative to overall 
sequence length. In this case the lcb quality parameter which represents the cost of the lcb divided by its length of lcb
 is set to the relatively low value of 25 to facilitate the detection of blocks within the sequences.  The higher the lcb 
 quality values will result in more stringent lcb determination and likely fewer local collinear blocks recovered.  The 
 second parameter within the argument \poyargument{annotate:(mauve)} sets the minimum lcb sequence coverage 
 at  30\% meaning that if total length of an input sequence is, for example,100, a minimum coverage of .30 would 
 require  a the total length of all lcbs to be at least 30. The default value of .01 or 1\% is sets the minimum length of a 
 given lcb  relative to the length of the entire sequence (e.g. 100 for a 10,000 nucleotide sequence). The maximum 
 length  allowed for an lcb in this example is set at 8\% of the length of the total sequence.
\item \texttt{transform(chromosome:(locus\_breakpoint:20,locus\_indel:\\(10,1.5),median:1,swap\_med:1)))}  The command 
\poycommand{transform} followed by the argument \poyargument{chromosome} specifies the conditions to be 
applied when calculating genome-level HTUs (medians). The argument \poyargument{chrom\_breakpoint:80} 
applies a breakpoint distance between chromosomes with the integer value determining the rearrangement cost. The 
argument \poyargument{chrom\_indel:15,1.5} specifies the indel costs for each entire chromosome, whereby the 
integer sets the gap opening cost and the float sets the gap extension cost.  The argument \poyargument{inversion:
20} applies an inversion distance between chromosome loci with the integer value determining the rearrangement 
cost. The argument \poyargument{locus\_indel:10,1.5} specifies the indel costs for the chromosomal segments, 
whereby the integer 10 sets the gap opening cost and the float 1.5 sets the gap extension cost.  The default values 
are applied to the \poyargument{median} and \poyargument{swap\_med} arguments to minimize the time require for 
these nested search options.   To more exhaustively perform these calculations trees generated from initial builds 
can be imported to the program and reevaluated with values greater than 1 designated for the \poyargument{median} 
and \poyargument{swap\_med} arguments
\item \texttt{build()} This commands begins the tree-building step of the search that generates by default 10 random-
addition trees.  It is highly recommended that a greater number of Wagner builds be implemented when analyzing 
data for purposes other than this demonstration.
\item \texttt{swap()} The \poycommand{swap} command specifies that each of the trees be subjected to an alternating 
SPR and TBR branch swapping routine (the default of \poy).
\item \texttt{select()} Upon completion of branch swapping, this command retains only optimal and topologically 
unique trees; all other trees are discarded from memory. 
\item \texttt{report("genome",diagnosis)}  The \poycommand{report} command in combination with a file name and the
 \poyargument{diagnosis} outputs the optimal median states and edge values to a specified file (\texttt{genome}). 
\item \texttt{report("genconsens",graphconsensus)}  The \poycommand{report} command in combination with a file 
name and the \poyargument{graphconsensus} generates a pdf strict consensus file of the trees generated 
(\texttt{genconsensus}). 
\item \texttt{exit()} This commands ends the \poy session.
\end{itemize}

%LIKELIHOOD TUTORIALS

%TUTORIAL STATIC
\section{Maximum likelihood analysis: Static}{\label{tutorial 9}}
This tutorial illustrates the analysis of static characters under the maximum likelihood criterion.  This analysis is of similar 
intensity to that of a search using the \emph{GTR} model in PhyML.  Maximum likelihood analyses can be 
computationally intensive, therefore parsimony alternatives to RAS under likelihood are provided.  
%Additionally, %throughout the tutorial, some familiarity with the terminology of likelihood analyses is assumed. 
%--removed by LC as this is assumed for all analyses, not just LK

\begin{verbatim}
(* Read in the data, specifying static characters and conduct a 
  parsimony search under direct optimization to generate a pool 
  of trees to be transformed to likelihood *)
read(prealigned:("9.fas",tcm:(1,0)))
build(100)
swap()
select()
perturb(transform(static_approx),iterations:50,ratchet:(0.2,3))
fuse(iterations:200,swap())
select()

(* Transform static data to likelihood characters, specifying 
  GTR + Gamma 4 model with empirical equilibrium frequencies   
  under traditional MAL *)
transform(likelihood:(gtr,rates:(gamma:(4)),priors:(estimate),
gap:(missing),mal))
swap(all,optimize:(model:(max_count:5),branch:join_region))
select(best:1)
report("9_LK.tre",trees:(branches))
report("9_LK.lkm",lkmodel)
exit()
\end{verbatim}

\begin{itemize}

\item \texttt{(* Read in the data, specifying static characters and conduct a parsimony search under direct 
optimization to generate a pool of trees to be transformed to likelihood *)} This first line of the 
script is a comment. While comments are optional and do not affect the analyses, they are useful for 
housekeeping purposes.
\item \texttt{read(prealigned:("9.fas",tcm:(1,0)))} This command imports the nucleotide sequence data file 
\texttt{9.fas} as prealigned characters. Note that the \poycommand{tcm} specified for this import is non-metric, 
but consistent with the assumption of indels as ``missing'' data. This tcm is included for consistency, as later 
likelihood swaps make equivalent assumptions.
\item \texttt{build(100)} This command generates 100 random-addition Wagner trees. A large number of 
independent starting points ensures that a large portion of tree space have been examined.
\item \texttt{swap()} \poycommand{swap} specifies that each of the 100 trees is subjected to alternating SPR and 
TBR branch swapping routines (the default of \poy). All trees found with the optimal score are stored in memory.
\item \texttt{select()} Upon completion of branch swapping, this command retains only optimal and topologically 
unique trees; all other trees are discarded from memory. 
\item \texttt{perturb(transform(static\_approx),iterations:50,ratchet:\\(0.2,3))} This command subjects the resulting 
trees to 50 rounds of parsimony ratchet, re-weighting 20\% of characters by a factor of 3.
\item \texttt{fuse(iterations:200,swap())} In this step, up to 200 swappings of subtrees identical in terminal 
composition but different in topology, are performed between pairs of best trees recovered in the previous step. 
This is another strategy for further exploration of tree space. Each resulting tree is further refined by local branch 
swapping under the default parameters of \poycommand{swap}.
\item \texttt{select()} Upon completion of branch swapping, this command retains only optimal and topologically
 unique trees; all other trees are discarded from memory.
 
\item \texttt{(* Transform the static data to likelihood characters, specify-\\ing GTR + Gamma 4 model with empirical 
equilibrium frequencies\\ under traditional MAL *)} A comment indicating the intent of the commands 
which follow.
\item \texttt{transform(likelihood:(gtr,rates:(gamma:(4)),priors:(estimate),\\gap:(missing),mal))} This command 
transforms the characters to static likelihood characters, using a \emph{GTR} + $\Gamma 4$ model, with empirical 
equilibrium frequencies under standard MAL. In this model, indels are treated as ``missing'' data, as for the 
preceding parsimony search. Several default values, such as \texttt{gap:(missing)}, and \texttt{gamma:(4)} are 
listed explicitly.
\item \texttt{swap(all,optimize:(model:(max\_count:5),branch:join\_region))} This command swaps the tree using 
subtree pruning and regrafting, specifying that iteration of likelihood model parameters will occur after every 5 
joins, and that iteration of branches will occur only in the area around the join point. These iteration 
specifications are less strict than the defaults for the purpose of speeding up the swapping and iteration. See the
likelihood section in heuristics for discussion. 
\item \texttt{select(best:1)} This command saves a single most optimal topology (with branch lengths) in 
memory. All other trees are purged.
\item \texttt{report("9\_LK.tre",trees:(branches))} This command outputs the topology, with branch lengths, in 
\texttt{.tre} format.
\item \texttt{report("9\_LK.lkm",lkmodel)} This command outputs the result of the likelihood analysis, which 
consists of the likelihood score, the variant of likelihood used, the tree length (sum of branch lengths), the values 
of the parameter estimates for the entries of the \textbf{P} and \textbf{Q} matrices, and the estimate of the value of 
the rate variation shape parameter.
\item \texttt{exit()} This commands ends the \poy session.
\end{itemize}

%TUTORIAL DYNAMIC

\section{Maximum likelihood analysis: Dynamic}{\label{tutorial10}}
The following script covers an analysis using dynamic Maximum Parsimonious Likelihood (MPL), and assumes 
that the reader has covered the previous tutorial, ``Maximum likelihood analysis: Static'' and is 
familiar with the intensity of the search strategy employed therein. 
%Additionally, throughout the tutorial, some familiarity with the terminology of likelihood 
%analyses is assumed. --removed by LC as this is assumed for all analyses, not just LK
\begin{verbatim}
(* Read in the data and conduct a parsimony search under direct 
   optimization (DO) to generate a pool of trees to be transformed 
   to likelihood *)
read("9.fas")
build(100)
swap()
select()
perturb(transform(static_approx),iterations:50,ratchet:(0.2,3))
fuse(iterations:200,swap())
select()

(* Transform the parsimony DO characters to dynamic MPL
   characters, specifying a GTR  model with empirical equilibrium
   frequencies and a shared nucleotide-indel substitution rate *)
transform(likelihood:(gtr,priors:(estimate),gap:(coupled),mpl))
swap(all,optimize:(model:(max_count:5)))
select(best:1)
report("9_LK.ia",ia:all)
report("9_LK_Ytree.tre",trees:(branches))
report("9_LK_Ymodel.lkm",lkmodel)
wipe()

(* Re-load the alignment/tree pair generated by dynamic MPL and
   re-diagnose as static MAL characters *)
read(prealigned:("9_LK.ia",tcm:(1,0)),"9_LK_Ytree.tre")
transform(likelihood:(gtr,rates:(gamma:(4)),priors:(estimate)))
report("9_LK_Ydyn_as_static.lkm",lkmodel)
wipe()

(* Re-load the implied alignment as a heuristic multiple sequence
    alignment and conduct a quick search under parsimony *)
read(prealigned:("9_LK.ia",tcm:(1,0)))
build(20)
swap()
select()
perturb(transform(static_approx),iterations:50,ratchet:(0.2,3))
select()

(* Transform the static characters to static MPL characters and 
  swap *)
transform(likelihood:(gtr,gap:(coupled),mpl))
swap(all,optimize:(model:(max_count:5)))
report("9_LK_YdynMPL_as_staticMPL.lkm",lkmodel)
report("9_LK_Ydyn_as_static.tre",trees:(branches))
exit()
\end{verbatim}

\begin{itemize}
\item \texttt{(* Read in the data and conduct a parsimony search under \\ direct optimization (DO) to generate a 
pool of trees to be \\ transformed to likelihood *)} This first line of the script is a comment. While comments are 
optional and do not affect the analyses, they are useful for separating different components of an analysis, 
especially if the script is long.  
\item \texttt{read("9.fas")}
This command imports the nucleotide sequence data file \texttt{9.fas} under the default behavior of treatment as 
dynamic characters. Note: unlike the previous tutorial, the characters are not imported as prealigned.
\item \texttt{build(100)} This command generates 100 random-addition Wagner trees. A large number of 
independent starting points ensures that a large portion of tree space have been examined.
\item \texttt{swap()} \poycommand{swap} specifies that each of the 100 trees is subjected to alternating SPR and 
TBR branch swapping routines (the default of \poy). All trees with the optimal score found are stored in memory.
\item \texttt{select()} Upon completion of branch swapping, this command retains only optimal and topologically 
unique trees; all other trees are discarded from memory. 
\item \texttt{perturb(transform(static\_approx),iterations:50,ratchet:\\(0.2,3))} This command subjects the resulting 
trees to 50 rounds of parsimony ratchet, re-weighting 20\% of characters by a factor of 3.
\item \texttt{fuse(iterations:200,swap())} In this step, up to 200 swappings of subtrees identical in terminal 
composition but different in topology, are performed between pairs of best trees recovered in the previous step. 
This is another strategy for further exploration of tree space. Each resulting tree is further refined by local branch
 swapping under the default parameters of \poycommand{swap}.
\item \texttt{select()} Upon completion of branch swapping, this command retains only optimal and topologically
 unique trees; all other trees are discarded from memory.

\item \texttt {(* Transform the parsimony DO characters to dynamic MPL \\characters, specifying a GTR model 
with empirical equili- \\ brium frequencies and a shared nucleotide-indel substitution rate *)} A comment indicating 
the intent of the commands which follow.
\item \texttt{transform(likelihood:(gtr,priors:(estimate),gap:(coupled),\\mpl))} This command transforms the 
characters to static likelihood characters, using a \emph{GTR} model, with empirical equilibrium frequencies under 
dynamic MPL. Note: Under dynamic MPL rate variation distribution is not enabled.
\item \texttt{swap(all,optimize:(model:(max\_count:5)))} This command swaps the tree using subtree pruning and 
regrafting, specifying that iteration of likelihood model parameters will occur after every 5 joins. These iteration 
specifications are less strict than the defaults for the purpose of speeding up the swapping. See the likelihood section
of heuristics for additional discussion. 
\item \texttt{select(best:1)} This command saves the single most optimal topology (with branch lengths) in 
memory. All other trees are purged.
\item \texttt{report("9\_LK.ia",ia:all)} This command saves the implied
    alignment generated by dynamic MPL as a file named \texttt{9\_LK.ia}.
\item \texttt{report("9\_LK\_Ytree.tre",trees:(branches))} This command saves the topology in memory with 
branch lengths as a \texttt{.tre} file named \texttt{9\_LK\_Ytree}.
\item \texttt{report("9\_LK\_Ymodel.lkm",lkmodel)} This command saves the parameter estimates generated 
by dynamic MPL as a file named \texttt{9\_LK\_Ymodel.lkm}.
\item \texttt{wipe()} This commands clears the memory.

\item \texttt{(* Re-load the alignment/tree pair generated by dynamic MPL and
   re-diagnose as static MAL characters *)} A comment indicating the intent of the commands 
which follow.
\item \texttt{read(prealigned:("9\_LK.ia",tcm:(1,0)),"9\_LK\_Ytree.tre")} Read in the tree/alignment 
combination produced by the preceding search under dynamic MPL as static characters.
\item \texttt{transform(likelihood:(gtr,rates:(gamma:(4)),priors:(estimate)))} Transform the static characters to MAL, 
using a \emph{GTR} + $\Gamma 4$ model and empirical equilibrium frequencies. POY will re-diagnose the topology 
using this model and optimize the values.
\item \texttt{report("9\_LK\_Ydyn\_as\_static.lkm",lkmodel)} Report the output of the likelihood analysis to
 the file \texttt{9\_LK\_Ydyn\_as\_static.lkm}.
\item \texttt{wipe()} This commands clears the memory.

\item \texttt{(* Re-load the implied alignment as a heuristic multiple \\
   sequence alignment and conduct a quick search under parsimony *)} A comment indicating the intent of the commands 
which follow.
\item \texttt{read(prealigned:("9\_LK.ia",tcm:(1,0)))} This command imports the nucleotide implied 
alignment \texttt{9\_LK.ia} as prealigned characters. Note that the \poycommand{tcm} specified for this import 
is non-metric, but consistent with the assumption of indels as ``missing'' data. This tcm is included for consistency, 
as later likelihood swaps make equivalent assumptions.
\item \texttt{build(20)} This command generates 20 random-addition Wagner trees. A large number of 
independent starting points ensures that a large portion of tree space have been examined.
\item \texttt{swap()} \poycommand{swap} specifies that each of the 20 trees is subjected to alternating SPR and 
TBR branch swapping routines (the default of \poy). All trees with the optimal score found are stored in memory.
\item \texttt{select()} Upon completion of branch swapping, this command retains only optimal and topologically 
unique trees; all other trees are discarded from memory. 
\item \texttt{perturb(transform(static\_approx),iterations:50,ratchet:\\(0.2,3))} This command subjects the resulting 
trees to 50 rounds of parsimony ratchet, re-weighting 20\% of characters by a factor of 3.
\item \texttt{select()} Upon completion of branch swapping, this command retains only optimal and topologically 
unique trees; all other trees are discarded from memory. 

\item \texttt{(* Transform the static characters to static MPL characters and swap *)} A comment indicating the 
intent of the commands which follow.
\item \texttt{transform(likelihood:(gtr,gap:(coupled),mpl)))} Transform the static parsimony characters to static 
MPL characters with a \emph{GTR} model, empirical equilibrium frequencies, and a coupled nucleotide-indel 
parameter.
\item \texttt{swap(all,optimize:(model:(max\_count:5)))} This command swaps the tree using subtree pruning and 
regrafting, specifying that iteration of likelihood model parameters will occur after every 5 joins. These iteration 
specifications are less strict than the defaults for the purpose of speeding up the swapping.
\item \texttt{report("9\_LK\_YdynMPL\_as\_staticMPL.lkm",lkmodel)} This command outputs the result of 
the likelihood analysis to the file \texttt{"9\_LK\_YdynMPL\_as\_staticMPL.lkm}.
\item \texttt{report("9\_LK\_Ydynamic\_as\_static.tre",trees:(branches))} This command outputs the topology, 
with branch lengths, in \texttt{.tre} format.
\item \texttt{exit()} This commands ends the \poy session.
\end{itemize}

%TUTORIAL HEURISTICS

\section{Maximum likelihood analyses: Heuristics}{\label{tutorial11}}
The following scripts cover analyses under the maximum likelihood criterion that employ heuristics of various intensities
and are of varying speed. The first section of this tutorial covers the analysis of static or prealigned data, the second, 
dynamic data. %Additionally, throughout the tutorial, some familiarity with the terminology of likelihood analyses is assumed.
%/ High Granularity / Thorough
\begin{verbatim}
(* TUTORIAL 11a: STATIC DATA; FAST ANALYSIS *) 

(* Read in the data, specifying static characters and conduct a 
  parsimony search under direct optimization to generate a pool 
  of trees to be transformed to likelihood *)
read(prealigned:("9.fas",tcm:(1,1)))
search(max_time:00:01:00)
transform(likelihood:(gtr,rates:(gamma:(4)),priors:(estimate),
gap:(missing),mal))
select(best:1)
swap(spr,all:5,optimize:(model:(threshold:0.33),branch:join_delta)
swap(spr,all:1)
report("faststat_LK.tre",trees:(branches))
report("faststat_LK.lkm",lkmodel)
exit()
\end{verbatim}

\begin{itemize}
\item \texttt{(* Read in the data, specifying static characters and conduct a parsimony search under direct optimization 
to generate a pool of trees to be transformed to likelihood *)} This first line of the script is a comment. While 
comments are optional and do not affect the analyses, they are useful for housekeeping purposes.
\item \texttt{read(prealigned:("9.fas",tcm:(1,1)))} This command imports the nucleotide sequence data file \texttt{9.fas} 
as prealigned characters.
\item \texttt{search(max\_time:00:01:00)} Specifies that the program will attempt as many builds, swaps, ratchets 
and tree fusings as possible within the specified time of one hour. All trees with the optimal score found are stored 
in memory. Note: as stated in Chapter 2, the execution time for the timed search is data dependent, 
therefore the time chosen here should not be taken as optimal for all data sets.
\item \texttt{transform(likelihood:(gtr,rates:(gamma:(4)),priors:(estimate),\\ gap:(missing),mal))} This command 
transforms the characters to likelihood characters, using a GTR + $\Gamma 4$ model, with empirical 
equilibrium frequencies under standard MAL. In this model, indels are treated as ``missing'' data, as for the 
preceding parsimony search. Several default values, such as \texttt{gap:(missing)}, and \texttt{gamma:(4)} are 
listed explicitly.
\item \texttt{swap(spr,all:5,optimize:(model:(threshold:0.33),branch:join\_region))} This command swaps the tree using 
subtree pruning and regrafting, with joins occurring within five branches of the break site. The model parameters are
optimized if the cost of the join under the current model is within 1.33 times the current best cost (proportion 0.33 worse). 
Only the branches along the path from the break to the new join location are optimized. 
\item \texttt{swap(spr,all:1)} This command swaps the tree using nearest-neighbor interchange (NNI), this time
optimizing all the model parameters and the branches after every join.
\item \texttt{report("faststat\_LK.tre",trees:(branches))} This command outputs the topology, with branch lengths, in 
\texttt{.tre} format.
\item \texttt{report("faststat\_LK.lkm",lkmodel)} This command outputs the result of the likelihood analysis, which 
consists of the likelihood score, the variant of likelihood used, the tree length (sum of branch lengths), the values 
of the parameter estimates for the entries of the \textbf{P} and \textbf{Q} matrices, and the estimate of the value of 
the rate variation shape parameter.
\item \texttt{exit()} This commands ends the \poy session.
\end{itemize}

\begin{verbatim}
(* TUTORIAL 11b: STATIC DATA; THOROUGH ANALYSIS *)

(* Read in the data, specifying static characters and conduct a 
  parsimony search under direct optimization to generate a pool 
  of trees to be transformed to likelihood  *)
read(prealigned:("9.fas",tcm:(1,1)))
search(max_time:00:02:00)
transform(likelihood:(gtr,rates:(gamma:(4)),priors:(estimate),
gap:(missing),mal))
swap(tbr,bfs,all)
report("thoroughstat_LK.tre",trees:(branches))
report("thoroughstat_LK.lkm",lkmodel)
exit()
\end{verbatim}

\begin{itemize}
\item \texttt{(* Read in the data and specify static characters *)} This first line of the script is a comment. While 
comments are optional and do not affect the analyses, they are useful for housekeeping purposes.
\item \texttt{read(prealigned:("9.fas",tcm:(1,1)))}
This command imports the nucleotide sequence data file \texttt{9.fas} as prealigned characters. 
\item \texttt{search(max\_time:00:02:00)} Specifies that the program will attempt as many builds, swaps, ratchets 
and tree fusings as possible within the specified time of two hours. All trees with the optimal score found are stored 
in memory. Note: as stated in Chapter 2, the execution time for the timed search is data dependent, 
therefore the time chosen here should not be taken as optimal for all data sets.
\item \texttt{transform(likelihood:(gtr,rates:(gamma:(4)),priors:(estimate),\\ gap:(missing),mal))} This command 
transforms the characters to likelihood characters, using a GTR + $\Gamma 4$ model, with empirical 
equilibrium frequencies under standard MAL. In this model, indels are treated as ``missing'' data, as for the 
preceding parsimony search. Several default values, such as \texttt{gap:(missing)}, and \texttt{gamma:(4)} are 
listed explicitly.
\item \texttt{swap(tbr,bfs,all))} This command swaps the tree using tree bisection and reconnection, using a breadth-first
search with no restrictions on the maximum distance for joins, and optimizing the model and branches after every join. 
\item \texttt{report("thoroughstat\_LK.tre",trees:(branches))} This command outputs the topology, with branch lengths, in 
\texttt{.tre} format.
\item \texttt{report("thoroughstat\_LK.lkm",lkmodel)} This command outputs the result of the likelihood analysis, which 
consists of the likelihood score, the variant of likelihood used, the tree length (sum of branch lengths), the values 
of the parameter estimates for the entries of the \textbf{P} and \textbf{Q} matrices, and the estimate of the value of 
the rate variation shape parameter.
\item \texttt{exit()} This commands ends the \poy session.
\end{itemize}

\begin{verbatim}
(* TUTORIAL 11c: DYNAMIC DATA: FAST ANALYSIS *)

(* Read in the data and conduct a parsimony search under direct 
  optimization to generate a pool of trees to be transformed to 
  likelihood.  Do not specify static characters *)
read("9.fas")
search(max_time:00:01:00)
transform(elikelihood:(gtr,mpl))
swap(spr,all:5,optimize:(model:(threshold:0.2)))
set(iterative:exact)
swap(spr,all:1)
report("fastdyn_LK.tre",trees:(branches))
report("fastdyn_LK.lkm",lkmodel)
exit()
\end{verbatim}

\begin{itemize}
\item \texttt{(* Read in the data and conduct a parsimony search under direct optimization to generate a pool of trees 
to be transformed to likelihood.  Do not specify static characters *)} This first line of the script is a comment. While 
comments are optional and do not affect the analyses, they are useful for housekeeping purposes.
\item \texttt{read("9.fas")} This command imports the nucleotide sequence data file \texttt{9.fas}. These data are 
not prealigned.
\item \texttt{search(max\_time:00:01:00)} Specifies that the program will attempt as many builds, swaps, ratchets 
and tree fusings as possible within the specified time of one hour. All trees with the optimal score found are stored 
in memory.  Note: as stated in Chapter 2, the execution time for the timed search is data dependent, 
therefore the time chosen here should not be taken as optimal for all data sets.
\item \texttt{transform(\hl{elikelihood}:(gtr,mpl)} This command transforms the characters to dynamic elikelihood (estimated 
likelihood) characters, using a GTR model, with empirical equilibrium frequencies under dynamic MPL.
\item \texttt{swap(spr,all:5,optimize:(model:(threshold:0.2)))} This command swaps the tree using subtree pruning and
regrafting, within a join distance of 5 from the break point and swaps the tree using subtree pruning and regrafting, 
with joins occurring within five branches of the break site. The model parameters are optimized if the cost of the join 
under the current model is within 1.2 times the current best cost (proportion 0.2 worse).
\item \texttt{set(iterative:exact)} Transform the data to regular dynamic MPL and optimize both the model and branches
on the given topologies in memory.
\item \texttt{swap(spr,all:1)} \hl{Conduct an NNI search on the topologies in memory, optimizing the model and branches after
each join.}
\item \texttt{report("fastdyn\_LK.tre",trees:(branches))} This command outputs the topology, with branch lengths, in 
\texttt{.tre} format.
\item \texttt{report("fastdyn\_LK.lkm",lkmodel)} This command outputs the result of the likelihood analysis, which 
consists of the likelihood score, the variant of likelihood used, the tree length (sum of branch lengths), the values 
of the parameter estimates for the entries of the \textbf{P} and \textbf{Q} matrices, and the estimate of the value of 
the rate variation shape parameter.
\item \texttt{exit()} This commands ends the \poy session.
\end{itemize}

\begin{verbatim}
(* TUTORIAL 11d: DYNAMIC DATA; THOROUGH ANALYSIS *)

(* Read in the data and conduct a parsimony search under direct 
  optimization to generate a pool of trees to be transformed to 
  likelihood.  Do not specify static characters *)
read("9.fas")
search(max_time:00:02:00)
transform(likelihood:(gtr,mpl))
swap(tbr,bfs,all)
report("thoroughdyn_LK.tre",trees:(branches))
report("thoroughdyn_LK.lkm",lkmodel)
exit()
\end{verbatim}

\begin{itemize}
\item \texttt{(* Read in the data and conduct a parsimony search under \\direct optimization to generate a pool of trees 
to be trans- \\formed to likelihood.  Do not specify static character *)} \\This first line of the script is a comment. While 
comments are optional and do not affect the analyses, they are useful for housekeeping purposes.
\item \texttt{read("9.fas")} This command imports the nucleotide sequence data file \texttt{9.fas}. These data are 
not prealigned. 
\item \texttt{search(max\_time:00:02:00)} Specifies that the program will attempt as many builds, swaps, ratchets 
and tree fusings as possible within the specified time of two hours. All trees with the optimal score found are stored 
in memory.  Note: as stated in Chapter 2, the execution time for the timed search is data dependent, 
therefore the time chosen here should not be taken as optimal for all data sets.
\item \texttt{transform(likelihood:(gtr,mpl)} This command transforms the characters to dynamic likelihood (MPL) 
characters, using a GTR model, with empirical equilibrium frequencies.
\item \texttt{swap(tbr,\hl{bfs},all)} This command swaps the tree using tree bisection and reconnection, under a breadth-first 
search. The model parameters and branch lengths are optimized after every join.
\item \texttt{report("thoroughdyn\_LK.tre",trees:(branches))} This command outputs the topology, with branch lengths, in 
\texttt{.tre} format.
\item \texttt{report("thoroughdyn\_LK.lkm",lkmodel)} This command outputs the result of the likelihood analysis, which 
consists of the likelihood score, the variant of likelihood used, the tree length (sum of branch lengths), the values 
of the parameter estimates for the entries of the \textbf{P} and \textbf{Q} matrices, and the estimate of the value of 
the rate variation shape parameter.
\item \texttt{exit()} This commands ends the \poy session.
\end{itemize}

\section{Maximum likelihood analysis: Morphology}{\label{tutorial 12}}
The following scripts cover analysis of morphological data under the maximum likelihood criterion. %Additionally, 
%throughout the tutorial, some familiarity with the terminology of likelihood analyses is assumed.

\begin{verbatim}
(* Read in data and report *)
read("31.ss")
report(data)

(* Conduct RAS and parsimony initial pass *)
build(30)
swap()
perturb(transform(static_approx),iterations:5,ratchet:(0.1,2))
perturb(transform(static_approx),iterations:5,ratchet:(0.3,3))
perturb(transform(static_approx),iterations:5,ratchet:(0.5,4))

(* Transform to likelihood characters *)
transform(likelihood:(jc69,alphabet:(4)))
swap(tbr,bfs:5,all:5,optimize:(model:(threshold:0.4),branch:
join_delta))
report("morphology_r=4.tre",trees:(branches))
report("morphology_r=4.lkm",lkmodel)

transform(likelihood:(jc69,alphabet:(min)))
swap(tbr,bfs:5,all:5,optimize:(model:(threshold:0.4),branch:
join_delta))
report("morphology_r=min.tre",trees:(branches))
report("morphology_r=min.lkm",lkmodel)
exit()
\end{verbatim}

\begin{itemize}
\item \texttt{(* Read in the data and report *)} This first line of the script is a comment. While comments are optional and do 
not affect the analyses, they are useful for housekeeping purposes.
\item \texttt{read("31.ss")}
This command imports the Hennig86 morphological data file \texttt{31.ss} as prealigned characters. 
\item \texttt{report(data)} This command generates a summary of the data properties. In reporting the data, it is possible to 
examine the data for the number of character states, i.e. the ``Range" column in the \poy Output window of the Interactive
Console. 
\item \texttt{build(30)} Build 30 random addition Wagner trees.
\item \texttt{swap()} Conduct alternating rounds of SPR and TBR until the scores stabilize.

\item \texttt{perturb(transform(static\_approx),iterations:5,ratchet:(0.1,2))} Conduct 5 iterations of parsimony ratchet,
reweighting 10 percent of the characters by a factor of 2.
\item \texttt{perturb(transform(static\_approx),iterations:5,ratchet:(0.3,3))} Conduct 5 iterations of parsimony ratchet,
reweighting 30 percent of the characters by a factor of 3.
\item \texttt{perturb(transform(static\_approx),iterations:5,ratchet:(0.5,4))} Conduct 5 iterations of parsimony ratchet,
reweighting 50 percent of the characters by a factor of 4.
\item \texttt{transform(likelihood:(jc69,alphabet:(4)))} Transform to likelihood characters under JC69/Neyman model, 
specifying the alphabet size (constant across all characters) to be $r=4$. The alphabet size equates to the number of 
character states identified in \texttt{report(data)}.
\item \texttt{swap(tbr,\hl{bfs}:5,all:5,optimize:(model:(threshold:0.4),branch:\\join\_delta))} Swap using TBR and a 
breadth-first search joining a maximum of five branches from the break site, optimizing the model parameters if the 
cost of the join under the current model is within 1.4 times the current best cost (proportion 0.4 worse), and optimizing 
only the branches along the path from the break to the new join location. 
\item \texttt{report("morphology\_r=4.tre",trees:(branches))} This command outputs the topology, with branch lengths, in 
\texttt{.tre} format.
\item \texttt{report("morphology\_r=4.lkm",lkmodel)} This command outputs the result of the likelihood analysis, which 
consists of the likelihood score, the variant of likelihood used, the tree length (sum of branch lengths), and the values 
of the parameter estimates for the entries of the \textbf{P} and \textbf{Q} matrices.

\item \texttt{transform(likelihood:(jc69,alphabet:(min)))} Transform to likelihood characters under JC69/Neyman model, specifying 
the alphabet size (constant across all characters) to be the minimum value that encompasses all character observations.
\item \texttt{swap(tbr,bfs:5,all:5,optimize:(model:(threshold:0.4),branch:\\join\_delta))} Swap using TBR and a breadth-first search
joining a maximum of five branches from the break site, optimizing the model parameters if the cost of the join under the 
current model is within 1.4 times the current best cost (proportion 0.4 worse), and optimizing only the branches along 
the path from the break to the new join location. 
\item \texttt{report("morphology\_r=min.tre",trees:(branches))} This command outputs the topology, with branch lengths, in 
\texttt{.tre} format.
\item \texttt{report("morphology\_r=min.lkm",lkmodel)} This command outputs the result of the likelihood analysis, which 
consists of the likelihood score, the variant of likelihood used, the tree length (sum of branch lengths), and the values 
of the parameter estimates for the entries of the \textbf{P} and \textbf{Q} matrices.
\item \texttt{exit()} This commands ends the \poy session.
\end{itemize}

\section{Maximum likelihood analysis: Partitions}{\label{tutorial 13}}
The following scripts cover analyses partitioned datasets under the maximum likelihood criterion. The first section of this 
tutorial covers the analysis of partitioned codons of protein coding sequences.  The second section covers the combined
analysis of both morphological and molecular data.
%The search intensity under likelihood presented below is moderate (see Tutorial \hl{X}: Likelihood Heuristics. --removed 
%by LC as there is no moderate tutorial
%Additionally, %throughout the tutorial, some familiarity with the terminology of likelihood analyses is assumed. 
%--removed by LC as this is assumed for all analyses, not just ML

\begin{verbatim}

(* TUTORIAL 13a: MAXIMUM LIKELIHOOD ANALYSIS; PARTITIONED CODONS *) 

(* Read in the data, specifying static characters. Conduct RAS 
   and swapping under parsimony *)
read(prealigned:("coleoptera_ND2.fasta",tcm:(1,1)))
search(max_time:00:01:00)

(* Partition the data *)
set(codon_partition:("coleoptera_codon",names:("coleoptera_
ND2.fasta")))

(* Transform to likelihood and complete analysis *)
transform(sets:("coleoptera_codon"),(likelihood:(gtr,rates:(gamma
(4)),priors:(estimate), gap:(missing),mal)))
swap(spr,all:5,optimize:(model:(threshold:0.33),branch:join_delta)
swap(spr,all:1)
report("coleoptera_codon_LK.tre",trees:(branches))
report("coleoptera_codon_LK.lkm",lkmodel)
exit()
\end{verbatim}

\begin{itemize}
\item \texttt{(* Read in the data, specifying static characters.  Conduct RAS and swapping under parsimony *)} 
This first line of the script is a comment. While comments are optional and do not affect the analyses, they are 
useful for housekeeping purposes.
\item \texttt{read(prealigned:("coleoptera\_ND2.fasta",tcm:(1,1)))} This command imports the nucleotide sequence 
data file \texttt{coleoptera\_ND2.fasta} as prealigned characters. 
\item \texttt{search(max\_time:00:01:00)} Specifies that the program will attempt as many builds, swaps, ratchets 
and tree fusings as possible within the specified time of one hour. All trees with the optimal score found are stored 
in memory.
\item \texttt{(* Partition the data *)} This is a comment.
\item \texttt{set(codon\_partition:("coleoptera\_codon",names:("coleoptera\_ND2.fasta")))} Specifies that the data be 
partitioned as ``coleoptera\_codon'' data, in which a partition is defined to include every third nucleotide position. 
This command is equivalent to the NEXUS partitioning commands
\\
\\
Begin SETS;\\
pos1 = 1-N /3;\\
pos2 = 2-N /3;\\
pos3 = 3-N /3;\\
END;\\

where $N$ is the aligned length of the static data. The data must begin at the first codon position and must be a multiple
of three.
\item \texttt{transform(sets:("coleoptera\_codon"),((likelihood:(gtr,rates:\\(gamma:(4)),priors:(estimate), gap:(missing),mal))} 
This command transforms each partition to a separate set of likelihood characters, using a GTR + $\Gamma 4$ model, with empirical 
equilibrium frequencies under standard MAL. In this model, indels are treated as ``missing'' data, as for the 
preceding parsimony search. Several default values, such as \texttt{gap:(missing)}, and \texttt{gamma:(4)} are 
listed explicitly. 
%The default summation method for POY partition contributions to the overall likelihood score is a simple summation 
%over partitions. But see the partitioning strategies of TREEFINDER, etc. 
% In the future, partitioning parameter optimization file used?
\item \texttt{swap(spr,all:5,optimize:(model:(threshold:1.33),branch:join\_region))} This command swaps the tree using 
subtree pruning and regrafting, with joins occurring within five branches of the break site. The model parameters are
optimized if the cost of the join under the current model is within 1.33 times the current best cost (proportion 0.33 worse). 
Only the branches along the path from the break to the new join location are optimized. 
\item \texttt{swap(spr,all:1)} This command swaps the tree using \hl{nearest-neighbor interchange (NNI)}, this time
optimizing all the model parameters and the branches after every join.
\item \texttt{report("coleoptera\_codon\_LK.tre",trees:(branches))} This command outputs the topology, with branch lengths in 
\texttt{.tre} format.
\item \texttt{report("coleoptera\_codon\_LK.lkm",lkmodel)} This command outputs the result of the likelihood analysis, which 
consists of the likelihood score, the variant of likelihood used, the tree length (sum of branch lengths), the values 
of the parameter estimates for the entries of the \textbf{P} and \textbf{Q} matrices, and the estimate of the value of 
the rate variation shape parameter.
\item \texttt{exit()} This commands ends the \poy session.
\end{itemize}

\begin{verbatim}

(* TUTORIAL 13b: MAXIMUM LIKELIHOOD ANALYSIS; COMBINED ANALYSIS 
  OF MORPHOLOGICAL AND MOLECULAR DATA *)

(* Read in morphology data and specify partition *)
read("9ia.ss")
set(partition:("morph",names:("9.fas")))
\end{verbatim}
%JD: Note--there is a bug here, as POY appears to rename the Hennig86 format as a fasta file. 
%Check report(data) in subsequent changesets.
%LC: I don't get this error when I read in 33.ss and set the partition
\begin{verbatim}
(* Read in prealigned molecular data; specify partition and 
  conduct RAS and swapping under parsimony *)
read(prealigned:("1_ia.fas",tcm:(1,0)))
set(partition:("molec",names:("1_ia.fas")))
search(max_time:00:01:00)

(* Transform sets to likelihood and complete analysis *)
transform(sets:("morph"),(likelihood:(jc69,alphabet:(min)))) 
transform(sets:("molec"),(likelihood:(gtr,rates:(gamma:(4)),
priors:(estimate), gap:(missing),mal)))
swap(tbr,bfs:10,all:10,optimize:(model:(threshold:1.6),branch:
join_region))
swap(spr,all:5)
report("combined_LK.tre",trees:(branches))
report("conbined_LK.lkm",lkmodel)
exit()
\end{verbatim}

\begin{itemize}
\item \texttt{(* Read in the morphology data and specify partition *)} This first line of the script is a comment. While 
comments are optional and do not affect the analyses, they are useful for housekeeping purposes.
\item \texttt{read(\hl{"9ia.ss"})} %There is no such file.  There is a 9.fas file, so a different file name for this new file should be 
%used to reduce confusion. --LC
This command imports the morphological data file \texttt{9ia.ss} in Hennig86 format. 
\item \texttt{set(partition:("morph",names:("9.fas")))}
This command defines a character partition called ``morph'' for all characters in the morphological dataset.

\item \texttt{(* Read in prealigned molecular data; specify partition and conduct RAS and swapping under parsimony *)}
This is a comment indicating the intent of the commands which follow.
\item \texttt{read(prealigned:("1\_ia.fas",tcm:(1,0)))} %There is no such file--LC
This command reads in the molecular dataset \texttt{1\_ia.fas} as prealigned data.
\item \texttt{set(partition:("molec",names:("1\_ia.fas")))}
This command defines a character partition called ``molec'' for all characters in the molecular dataset.
\item \texttt{search(max\_time:00:01:00)} Specifies that the program will attempt as many builds, swaps, ratchets 
and tree fusings as possible within the specified time of one hour. All trees with the optimal score found are stored 
in memory. Note: as stated in Chapter 2, the execution time for the timed search is data dependent, 
therefore the time chosen here should not be taken as optimal for all data sets.

\item \texttt{transform(sets:("morph"),(likelihood:(jc69,alphabet:(min))))} %This command produces a NOT FOUND error.
This command specifies a partition-specific JC69 model with minimal Neyman alphabet size be applied to the 
partition ``morph.''
\item \texttt{transform(sets:("molec"),(likelihood:(gtr,rates:(gamma:(4)),\\priors:(estimate), gap:(missing),mal)))} 
This command specifies a partition-specific GTR + $\Gamma 4$ model, with empirical equilibrium frequencies 
under standard MAL be applied to the partition ``molec.'' 
\item \texttt{swap(tbr,bfs:10,all:10,optimize:(model:(threshold:1.6),\\branch:join\_region))}
This command swaps all trees in memory using tree bisection and reconnection, specifying a breadth-first search with join
points up to 10 branches from the break site. The model parameters are optimized if the cost of the join under the 
current model is within 1.6 times the current best cost (proportion 0.6 worse). Only the branches along the path from the 
break to the new join location are optimized.  
\item \texttt{swap(spr,all:5))} This command swaps the tree using subtree pruning and regrafting, using a depth-first
search with joins up to five branches from the break site, and optimizing the model and branches after every join. 
\item \texttt{report("combined\_LK.tre",trees:(branches))} This command outputs the topology, with branch lengths, in 
\texttt{.tre} format.
\item \texttt{report("combined\_LK.lkm",lkmodel)} This command outputs the result of the likelihood analysis, which 
consists of the likelihood score, the variant of likelihood used, the tree length (sum of branch lengths), the values 
of the parameter estimates for the entries of the \textbf{P} and \textbf{Q} matrices, and the estimate of the value of 
the rate variation shape parameter.
\item \texttt{exit()} This commands ends the \poy session.
\end{itemize}
