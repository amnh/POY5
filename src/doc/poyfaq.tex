This FAQ is supplementary documentation that aims to answer the
most frequently poised questions to the \poy developers and on the
\poy google groups.

\renewcommand{\cftdotsep}{\cftnodots} % This gets rid of the dots locally
\cftpagenumbersoff{questions} % This turns page numbers off for just the questions
\listofquestions
\newpage


%\section{Getting to know \poy}
\question{What does POY stand for?}
{POY is a meta-acronym, which comes from an older program YAPP (Yet
Another Phylogeny Program) that was written in \texttt{C}.  This
program, which was an extension of \texttt{MALIGN}, was the first
designed around direct optimization.  This program was rewritten
in Ocaml (Ocaml YAPP), which was shortened to OY.  The subsequent
parallelization of this program yielded POY.}

\question{I would like to visualize the implied alignment generated from 
my analysis, how do I do this?} 
{The implied alignment can be exported from \poy in two ways:  \\

The first is to export the implied alignment, using the command report.
The implied alignment will be output in FASTA format 
(see \nccross{implied\_alignments}{impliedalignment}).  \\

The second is to transform using static\_approx and report using
the phastwinclad option (see \nccross{phastwinclad}{phastwinclad}). 
This will produce a file Hennig86 format.\\
\\
These files can subsequently be imported into other programs such as 
Winclada or Mesquite, for visualization.}

\question{Looking at the implied alignment generated from my \poy
analysis, it looks very `gappy'. Why?} 
{POY is not, nor has it ever been an alignment program. Non-homologous, 
independent insertions are assigned their own columns with \poy, hence, 
the number of columns will expand with the number of insertions.}

%\section{Errors}
\question{I encountered a problem while running an analysis that I
think might be a bug in the program, how should I report this?}
{All error and bug reports should be made directly to the \poy Mail
Group. Before posting to this group, it is advised that the user
search the history of previously posed questions, to make sure that
it has not been answered previously.  When reported to the Mail
Group, users should include the following information: \\
\\
-- What steps will reproduce the problem; \\
-- What is the expected output and what do you see instead; \\
-- What version of the program are you using and on which operating system?}

\question{My script won't run and I don't know where I went wrong.
What should I do?}
{If a script won't run, the first thing to do is to check that there
are no hidden characters in the script file.  When constructing a
script or a transformation cost matrix, it is important to do so
in a text editor such as Notepad (for Windows), TextEdit (for Mac),
or Nano (for Linux). Generating these files in a word processing
application such as Microsoft Word may lead to the insertion of
hidden characters, which can result in an error.

Secondly, the user is advised to check the log.  If none was set,
the user is advised to do so and rerun the script.  The log will
give some indication as to which errors were encountered, or which
warnings were issued.}

\question{When I run \poy in parallel, I get multiple, identical
outputs to the screen, why?}
{It is likely that \poy was not properly compiled
in parallel. You should check the \texttt{make} options.}

\question{In running an analysis of custom alphabet characters,
with the characters being transformed to \poyargument {level} 5, I
got the following error "\texttt{seg fault:11}".  In a previous
analysis, the characters were transformed to \poyargument {level}
4, and that worked without issue.  What's wrong?} 
{This is most likely an `out of memory' error and is system specific 
and difficult to predict. This is beyond the control of the program. Storage and
set up time increase combinatorially with level number.}

\question{In running an analysis using a *.ss or Hennig86 file, I encountered 
a `syntax' error, however, the characters continued to load and the analysis seemed
to run.  Is something wrong?}
{Syntax errors are of the form: 
\\
Error: Syntax error\\
Error: Unrecognized command between characters 93 and 94 \\
Information: The file Fly.ss defines 9 static homology characters, \\
0 unaligned sequences, and 0 trees, containing 5 taxa\\
\\
Hennig86 or *.ss files, have no formal definition.  \poy does its
best to parse the file properly, but it is incumbent upon the user
to make sure that the data was read/parsed correctly.  If an error
such as this is encountered, the user should report the data
(\poyargument{report(data)}) to verify.  A similar caution should
be taken with NEXUS files, as very often files that have been
generated and exported from other programs are not in the correct
NEXUS format.}

%\section{SAQs: Stoopid Ass Questions} 
\question{My FASTA file contains sequences that are of poor 
`quality', especially in the 5 prime and 3 prime regions of the sequences.
How should these data files be prepared for analysis in \poy?}
{In cases such as this, partitioning the data is {\bf highly} recommended.  
Partitioning or fragmenting the data can help to ameliorate the effects of
poor sequences or missing data.  Moreover, when a data file contains
 sequences that were downloaded from a data base such as GenBank, 
 very often there is a lack of overlap of many of the sequences,
as different studies may have utilized different priming regions. 
How best to `chop' up your data is discussed in the \poy Heuristics chapter.}

\question{I read in a tree that was generated from a previous analysis, 
however, the cost reported in the output window of the \texttt {Interactive 
Console} is different, why is this the case?} 
{If you are reading in a tree generated from a previous analysis it is 
important to make sure that the same transformation cost matrix has 
been applied to the data. In addition, the tree must be fully resolved, 
otherwise it will be resolved arbitrarily.}

\question{Having run an analysis in \poy, I imported my tree file into 
TNT, but the tree costs are different.  Is this an error?} 
{When importing \poy tree files into another program, such as TNT, 
it is important to mirror the same `conditions' as to those in \poy
during the time of analysis, i.e. same cost matrix, gaps treated as a 
fifth state. The correct data file associated with this tree file, must also 
be imported---this corresponds to the implied alignment that is 
associated with this tree. In addition, the tree must be fully resolved, 
otherwise it will be resolved arbitrarily.}

\question{In trying to calculate Jackknife support values, I believe
that all the values are inflated for the resulting tree.  Why?}
{Although it is possible to calculate Jackknife and Bootstrap support
values for trees constructed using dynamic homology characters, it
is not recommended since resampling of dynamic characters
occurs at the fragment, rather than nucleotide, level. (Of course
this is a mute point if the dataset consists of a multitude of fragments.)
Consequently, the bootstrap and jackknife support values calculated for dynamic
characters are not directly comparable to those calculated based
on static character matrices. In order to perform character sampling
at the level of individual nucleotides, the dynamic characters {\bf
must} be transformed into static characters using \poyargument
{static\_approx} argument of the command transform (Section 3.3.26)
prior to executing calculate support.  The static\_approx is conditioned
or based on that tree.}

\question{Why is my prealigned data not treated as prealigned?} 
{By default, upon importing prealigned sequence data, the gaps are
removed and the sequences are treated as dynamic homology characters.
To preserve the alignment the data must be imported using the
\poyargument{prealigned} argument of the command \poycommand{read}.
Unless specified using the \poyargument {prealigned}, data that is
read by the program is UNALIGNED and the gaps are stripped from the
data file.}

\question{Is it possible to exclude certain terminals from an analysis?}
{The exclusion of terminals (or for that matter, characters) is easily
achieved by selecting, with the use of the identifiers.  For example
\texttt{select(terminals,not files:("Taxa\_removed.txt"))} will exclude all the 
taxa that are included in this file.  This is the inverse of 
\texttt{select(terminals,\\ files:("Taxa\_keep.txt"))}. Alternatively, if the user does not wish 
to generate a terminals file, the taxon names can be specified using
the \texttt{not names} identifier, 
e.g. \texttt{select(terminals,not names:("Taxon1","Taxon4"))}.}

\question{Is it possible to report parsimony branch lengths in \poy?}
{Yes it is. This is achieved by reporting \texttt{branches}, along with 
the \texttt{trees}, and specifying the collapse mode. For example
\texttt{report("Run1.tre", trees:(branches:single))} will report a
tree to the file \texttt{Run1.tre} with the parsimony branch lengths 
included.  The argument \texttt{single} specifies that zero length
branches will be collapsed.}

\question{I would like to import trees from an earlier run, at what
stage of the analysis should this be performed?} 
{When running a script that includes reading in trees from a previous analysis,
these trees {\bf must} be read in {\bf after} the build stage.  If
the trees are read in before the build they will be replaced by the
trees generated during the build.}

\question{Why is the root in the diagnosis file that I reported at
the end of my analysis, not the same as the root that I \poyargument{set}?}
{This is because the tree length heuristics may be based on an
alternate rooting scheme than that used for the \poyargument{newick}
or \poyargument {graphic trees} output.}

\question{What are "Numerical.linesearch; Very large slope in
optimization function" warnings?} 
{These messages normally appear in the Dynamic likelihood routines. 
They indicate that the gradient
of the parameters for the current data-set has a large absolute
slope and the numerical routine may not converge properly. Normally,
because of multiple passes of the optimization routine, we easily
break out of these regions and the routine will stabilize.

Under static likelihood data this warning message is rarely seen
and may be an issue. One should rediagnose the tree for stabilization
of the parameters of the model.}

\question{What does the "Numerical.brent; hit max number of
iterations" warning mean?} 
{This happens when the numerical routine
does not converge in a maximum number of steps. This is usually an
acceptable situation to happen as more optimization rounds will
likely occur over the data.}

\question{Why are likelihood scores worse/different than other
applications?} 
{There are a number of issues to consider. If the
values are close under the same model of evolution then numerical
issues due to finite precision arithmetic of decimal numbers can
cause slight rounding errors in an analysis to build up. Although
we use standard techniques to limit the accumulation of these errors,
they inevitably occur and small differences in likelihood scores
are absolutely normal and should not be a concern. These differences even
occur within the same application run on different architectures
and compiler options.

If the model is not hierarchical then comparisons are not relevant.
Unfortunately, the number of states in the model of evolution
matters, as well as cost assignments of Maximum Parsimonious
Likelihood (MPL) and Maximum Average Likelihood (MAL). Thus, a four
state model of evolutions' likelihood score cannot be compared
directly with a five state model. There is added cost that result
from the probability of an additional state in an analysis.

One should also check that the \poycommand{exhaustive} option is
set for the optimization routines.  It is set by default but ensure
that, if it changed somewhere in the script, one sets it back--especially 
if one plans to do application comparisons.}
