This FAQ is supplementary documentation that aims to answer the most frequently poised questions to 
the \poy developers and on the \poy google groups.

\section{Getting to know \poy}
Q. What does POY  stand for?\\
\\
A. POY is a meta-acronym, which comes from an older program YAPP (Yet Another Phylogeny Program) that 
was written in \texttt{C}.  This program, which was an extension of \texttt{MALIGN}, was the first designed around direct 
optimization.  This program was rewritten in Ocaml (Ocaml YAPP), which was shortened to OY.  
The subsequent parallelization of this program yielded POY.\\
\\
\\
Q. Looking at the implied alignment generated from my \poy analysis, it looks very gappy.  Why?\\
\\
A. POY is not, nor has it ever been an alignment program.\\
\\
\\
\section{Errors}
Q. I encountered a problem while running an analysis that I think might be a bug in the program, how should 
I report this?\\
\\
A. All error and bug reports should be made directly to the \poy Mail Group, which can be located at
\url{https://groups.google.com/forum/#!forum/poy4}. When reported to the Mail Group, users
should include the following information: what steps will reproduce the problem; what is the expected output and 
what do you see instead;  what version of the product are you using and on which operating system?\\
\\
\\
Q. ?
\\
A. Can't have prealigned with \texttt{affine} gap cost, only possible with dynamic.\\
\\
\\
Q. When I run \poy in parallel, I get multiple, identical outputs to the screen, why?\\
\\
A. The only explanation for this is that \poy was not properly compiled in parallel. You should check the \texttt{make}
options.\\
\\
\\
Q. My script won't run...\\
\\
A. If a script won't run, the first thing to do is to check that there are no hidden characters in the script file.
When constructing a script or a transformation cost matrix, it is important to do so in a text editor 
such as Notepad (for Windows), TextEdit (for Mac), or Nano (for Linux). Generating these files in 
a word processing application such as Microsoft Word may lead to the insertion of hidden 
 characters, which will result in an error.\\

\section{SAQs}
Q. I read in a tree that was generated from a previous analysis, however, the cost reported in the output window
of the \texttt {Interactive Console } is different, why is this the case?\\
\\
A. If you are reading in a tree generated from a previous analysis it is important to make sure that the same 
transformation cost matrix has been applied to the data.\\
\\
\\
Q. In trying to calculate Jackknife support values, I believe that all the values are inflated for the resulting tree.
Why?\\
\\
A. Although it is possible to calculate Jackknife and Bootstrap support values for trees constructed using dynamic 
homology characters, it is recommended against doing so as resampling of dynamic characters occurs at the fragment, 
rather than nucleotide, level. Consequently, the bootstrap and jackknife support values calculated for dynamic 
characters are not directly comparable to those calculated based on static character matrices. In order to perform 
character sampling at the level of individual nucleotides, the dynamic characters {\bf must} be transformed into 
static characters using \poyargument {static\_approx} argument of the command transform (Section 3.3.26) prior 
to executing calculate support.\\
\\
Q. Why is my prealigned data not treated as prealigned?\\
\\
A.  By default, upon importing prealigned sequence data, all the gaps are removed and the sequences are 
treated as dynamic homology characters. To preserve the alignment the data must be imported using the
\poyargument{prealigned} argument of the command \poycommand{read}.  Unless specified using the 
\poyargument {prealigned}, data that is read by the program is UNALIGNED and the gaps are stripped 
from the data file!!!!\\
\\
Q. I would like to import trees from an earlier run, at what stage of the analysis should this be performed?\\
\\
A. When running a script that includes reading in trees from a previous analysis, these trees {\bf must} be read 
in {\bf after} the build stage.  If the trees are read in before the build they will be replaced by the trees 
generated during the build.\\
\\
\\
Q.\\
\\
A.\\
\\